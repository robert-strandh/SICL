\chapter{Cross compilation}
\label{chap-cross-compilation}

While it may seem obvious and straightforward (though perhaps not
easy) to write a cross compiler for \cl{}, there are some minor issues
that have to be considered. 

First, let us define \emph{cross compiler} to mean a compiler that is
entirely written using some \cl{} system called the \emph{host}, and
which generates a \emph{fasl} file for some other \cl{} system called
the \emph{target}.  When we say \emph{entirely}, we mean that it is
the \texttt{read} function of the host that is used.  It would be
possible, of course to use the \emph{read} function of the target, but
the difference is small.

Clearly, since we are talking about the \emph{file compiler} we face
the same restrictions concerning literal objects as a native file
compiler does.%-----------------------------------------------------
\footnote{See Section 3.2.4 in the HyperSpec.} %--------------------
In addition, though, there are some restrictions due to differences
between systems that the HyperSpec explicitly allows.  

The most important such restriction has to do with floating-point
numbers.  If (say) the host allows for fewer types of floating-point
numbers, then \texttt{read} will not accurately represent the source
code as the native file compiler for the target would.  Code to be
compiled by the cross compiler must therefore either avoid
floating-point literals altogether, or instead use some expression to
create it and make sure that the expression is not evaluated until
load time. 

The other restriction has to do with \emph{potential numbers} which
different systems may define differently.  The easy solution is to
avoid potential numbers in source code.  This should not be hard to
do. 


