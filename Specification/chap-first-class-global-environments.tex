\chapter{First-class global environments}

Most \commonlisp{} implementations do not have an explicit
(first-class) representation for the global environments, and most
\commonlisp{} implementations have a single global environment.  This
environment is \emph{spread out} all over the system.  The
\texttt{symbol} objects of such an implementation may contain a slot
for a function named by that symbol and perhaps a slot for the global
value of a variable named by that symbol.  The mapping from class
names to class objects can also be represented as a slot in the symbol
object, but is more likely represented separately in a hash table.
Similarly, method combinations, \texttt{setf} expanders, and other
mappings from names to objects may be represented in yet another way.

\sysname{} uses \emph{first-class global environments} for all these
mappings.  There are several reasons for this way of representing
environments:

\begin{itemize}
\item It allows for multiple global environments, which in turn
  provides solutions to a number of common problems.  For example,
  several different versions of a system can be loaded simultaneously,
  provided that each version is loaded into a separate environment.
\item It becomes possible (and easy) to distinguish the three types of
  global environments that the \commonlisp{} standard allows (and
  sometimes encourages), namely run-time environments, evaluation
  environments, and compilation environments.
\item During bootstrapping of \sysname{} using a \emph{host}
  \commonlisp{} system, \sysname{} objects and mappings can be
  isolated from host objects and mappings by the use of first-class
  environments.
\end{itemize}

First-class environments also allow for other features such as
\emph{sandboxing} that are not directly used by \sysname{}.

This module has been extracted to a separate repository by the name of
\clostrum{}%
\footnote{https://github.com/s-expressionists/Clostrum} and it
provides several important features:

\begin{itemize}
\item The run-time environment is operated upon by a number of generic
  functions that mimic standard \commonlisp{} functions such as
  \texttt{fdefinition}, \texttt{macro-function}, etc., except that
  these functions have an additional parameter named \emph{client}.
  Client code can provide primary or auxiliary methods that specialize
  to some client-provided class, so as to override or extend the
  behavior of the default methods provided by \clostrum{}.
\item The evaluation environment is similar to the run-time
  environment, but also provides trampoline methods that calls the
  same generic function in the run-time environment whenever some
  object is not found in the evaluation environment.
\item The compilation environment is different, and can be used to
  hold imple\-mentation-specific information provided by the compiler
  during the compilation process, and queried by other parts of the
  compiler so as to provide feedback to the user in the form of errors
  or warnings.
\end{itemize}

Furthermore, \clostrum{} environments can be configured in multiple
\emph{layers}, so as to allow for \emph{incremental deltas}.  This
feature could be used in a text editor for parsing the contents of a
text buffer containing \commonlisp{} code.  When the buffer contents
is modified, side effects on the global environment by the old
contents can be undone by simply abandoning the upper layers while
preserving the environment contents associated with the code preceding
the modification.
