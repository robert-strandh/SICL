\chapter{Debugger}
\label{chap-debugger}

Part of the reason for \sysname{} is to have a system that provides
excellent debugging facilities for the programmer.  The kind of
debugger we plan to support is described in a separate repository.%
\footnote{See https://github.com/robert-strandh/Clordane}  In this
chapter, we describe only the support that \sysname{} contains in
order to make such a debugger possible.

Each function contains two versions of the code, called the
\emph{normal} version and the \emph{debugging} version.%
\footnote{This idea was suggested by Michael Raskin.}
Calls from the normal version of a caller are made to the normal version
of callees, and calls from the debugging version of a caller are made to
the debugging version of callees.

The normal version is used when the thread is not run under the
control of the debugger, so this version does not contain any code for
communicating with the debugger.  Furthermore, this version is highly
optimized.  In particular, variables that occur in the source code may
have been eliminated by various optimization passes.

In the debugging version of a function, the compiler associates
additional code with the program point immediately preceding the
evaluation of a form and with the program point immediately following
the evaluation of a form.  This code consists of a single call to a
global function, the name of which is yet to be determined.  For now,
let us call it \texttt{breakpoint}.  It is a function of no arguments.
However, recall that functions take implicit arguments, and the
important one for this discussion is the \emph{call-site information}
implicit argument.  It contains information about the call site such
as locations of live objects, and information for the garbage
collector.  The call is a \emph{named call}, and the code generated
such a call is just an unconditional \texttt{jump} instruction.  The
\emph{call-site manager} is responsible for allocating a
\emph{trampoline snippet} to mediate between the call-site and the
callee, and it then modifies the target field of the instruction to
contain the address of the snippet.  The call-site manager handles
calls to the \texttt{breakpoint} function differently.  By default, no
snippet is allocated, and the target field of the unconditional
\texttt{jump} is just the next instruction.  However, when a
breakpoint is set at some program point, the call-site manager
generates a snippet that will actually call the \texttt{breakpoint}
function, providing it with call-site information in the implicit
argument.  The \texttt{breakpoint} starts by consulting the
\texttt{thread} object, to determine whether this particular
application thread has a breakpoint at this particular program point.
If it does, the \texttt{breakpoint} function then accesses the
relevant information about live variables etc., collects it in a form
acceptable to the debugger, and then calls the debugger with this
information.  Typically, the debugger will then block the thread and
interact with the user for further actions.

The debugging version does not have optimizations applied to it that
may make debugging harder.  Lexical variables that appear in source
code may be kept, or code may be included that can compute their
values from the lexical variables that \emph{are} kept, for the
duration of their scope.

%% For each possible breakpoint, the system must keep a description of
%% the lexical environment.  This includes mappings from variable names
%% to registers or stack locations, information about liveness of
%% registers and stack location, how a variable is stored in a location
%% (immediate value, pointer, with or without type tag, etc).

