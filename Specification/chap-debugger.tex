\chapter{Debugger}
\label{chap-debugger}

Part of the reason for SICL is to have a system that provides
excellent debugging facilities for the programmer.  We imagine that a
thread will be debugged by a debugger running in a different thread.
The debugger should be able to set breakpoints before and after
expressions, and at the beginning or the end of a function.  
We imagine this being done by a conditional branch at strategic points
in the code, perhaps implemented as a ``skip the next instruction if
condition is false'' instruction, if that should turn out to be faster
than a normal conditional branch.  The condition tests a flag in the
current thread.  If the flag is set, then a call is made to determine
whether there is a breakpoint at this place.  If not, it returns and
execution continues as usual.  If there is a breakpoint, then
execution stops and the debugger thread is given the possibility to
inspect the state of the debugged thread. 

For the highest debug level, the conditional branch should be
generated before and after each expression, including each variable
reference.  This can be a bit costly for local variables because it
would slow down execution significantly.  Lower debug levels may
generate the conditional branch only when the cost of the branch is
negligible compared to the cost of evaluating the expression. 

In order not to slow down the execution too much, there should be a
quick test to determine that there is no breakpoint at a particular
place in the code.  We still have to think of ways of doing that.  One
might imagine a conservative test that is quick but possibly not
entirely accurate, perhaps based on intervals of values of the program
counter (say, take PC, shift it right by some number of bits, check a
bitvector to see if the corresponding bit is set, if not, there is no
breakpoint here). 

For each possible breakpoint, the system must keep a description of
the lexical environment.  This includes mappings from variable names
to registers or stack locations, information about liveness of
registers and stack location, how a variable is stored in a location
(immediate value, pointer, with or without type tag, etc). 

