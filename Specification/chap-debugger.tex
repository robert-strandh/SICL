\chapter{Debugger}
\label{chap-debugger}

Part of the reason for SICL is to have a system that provides
excellent debugging facilities for the programmer.  In this chapter,
we describe how we imagine such facilities would be accomplished.

Under normal circumstances, when a function is compiled, no matter
what the \texttt{debug} settings are, the compiler will generate two
versions of the body of the function:

\begin{itemize}
\item The \emph{normal} version of the body will be fully optimized.
  The compiler might not preserve the values of variables present in
  the source code because of various transformations such as
  \emph{liveness analysis} and \emph{strength reduction}.  Boolean
  values may never be created if they are part the \emph{test} in a
  conditional form.
\item The \emph{debugging} version of the body will have a very direct
  correspondence with the source code.  All variables present in the
  source will have places allocated for them as long as they are in
  scope in the source code.  Furthermore, before and after each source
  form, a call is inserted to an \emph{auxiliary function} which takes
  the current value of the program counter as an argument.  The role
  of the auxiliary function is describe below.
\end{itemize}

The \emph{function object} will contain two entry points that can be
swapped.  The \emph{normal entry point} executes the normal version of
the body described above.

The \emph{debugging entry point} works as follows: The first action of
the entry point is to access the \emph{current thread}, which is
located at the bottom of the dynamic environment, as the value of a
special variable.  The thread contains a slot that indicates whether
this thread is being debugged.  If the value of the slot is
\emph{false}, then the normal version of the body is executed as with
the normal entry point.  Thus, the only overhead of using the
debugging entry point in a thread that is not being debugged is the
test to determine whether the current thread is being debugged.  If
the value of the slot is \emph{true}, another test is made to see
whether \emph{the current function} is being debugged.  This test is a
bit slower, because it consists of searching a list contained in
another slot in the thread object.  That list contains the functions
that are currently being debugged.  If this function is not currently
being debugged, again the normal version of the body is executed.
Thus, the overhead of a function that is not being debugged in a
thread that is being debugged is searching a list, probably containing
a few dozen entries at most.

The two entry points are swapped so that the debugging entry point is
used if there exists a thread that has flagged the function as being
debugged.  When no thread has flagged the function as being debugged,
the normal entry point is used.

The \emph{auxiliary function} that is called before and after the
evaluation of each form in the debugging version of the function body
searches the thread object for some \emph{action} to be taken at this
particular point in the execution.  That action can be to suspend the
thread and give control to the debugging thread (as in the case of a
breakpoint), display some information (as in the case of the function
being traced), test the value of a local variable (as in the case of a
watch point), or anything else that the debugger interface allows.

Clearly, a function being flagged by some thread as being debugged is
going to be significantly slower \emph{when executing in one of the
  threads having flagged it}.  However a thread would flag a function
as being debugged only if it expects to set breakpoints in it, or
extract some other information, so the slower execution will probably
be dwarfed by the time it takes to process the desired information.

%% Part of the reason for SICL is to have a system that provides
%% excellent debugging facilities for the programmer.  We imagine that a
%% thread will be debugged by a debugger running in a different thread.
%% The debugger should be able to set breakpoints before and after
%% expressions, and at the beginning or the end of a function.
%% We imagine this being done by a conditional branch at strategic points
%% in the code, perhaps implemented as a ``skip the next instruction if
%% condition is false'' instruction, if that should turn out to be faster
%% than a normal conditional branch.  The condition tests a flag in the
%% current thread.  If the flag is set, then a call is made to determine
%% whether there is a breakpoint at this place.  If not, it returns and
%% execution continues as usual.  If there is a breakpoint, then
%% execution stops and the debugger thread is given the possibility to
%% inspect the state of the debugged thread.

%% For the highest debug level, the conditional branch should be
%% generated before and after each expression, including each variable
%% reference.  This can be a bit costly for local variables because it
%% would slow down execution significantly.  Lower debug levels may
%% generate the conditional branch only when the cost of the branch is
%% negligible compared to the cost of evaluating the expression.

%% In order not to slow down the execution too much, there should be a
%% quick test to determine that there is no breakpoint at a particular
%% place in the code.  We still have to think of ways of doing that.  One
%% might imagine a conservative test that is quick but possibly not
%% entirely accurate, perhaps based on intervals of values of the program
%% counter (say, take PC, shift it right by some number of bits, check a
%% bitvector to see if the corresponding bit is set, if not, there is no
%% breakpoint here).

%% For each possible breakpoint, the system must keep a description of
%% the lexical environment.  This includes mappings from variable names
%% to registers or stack locations, information about liveness of
%% registers and stack location, how a variable is stored in a location
%% (immediate value, pointer, with or without type tag, etc).

