\chapter{Lexical compile-time environments}
\label{chap-lexical-compile-time-environments}

A \commonlisp{} compiler must maintain a compile-time environment that
evolves during the compilations process.  Nested constructs such as
\texttt{let} and \texttt{labels} create new entries that are valid as
long as the construct is in scope. The module named \trucler{}%
\footnote{https://github.com/s-expressionists/Trucler} provides this
functionality.

\trucler{} provides a \clos{} \emph{protocol} for querying and
augmenting lexical environments.  The functionality provide has the
same purpose as the functions documented in section 8.5 of the second
edition of ``Common Lisp the language'' by Guy Steele
\cite{Steele:1990:CLL:95411}.  However the functionality documented
there is insufficient (it does not contain information introduced by
\texttt{block} and \texttt{tagbody} special operators, for instance),
and not extensible (because it relies on a fixed number of return
values).  In contrast, the protocol provided by \trucler{} allows
client code to customize the behavior.  For that purpose, \trucler{}
methods have a \emph{client} parameter that is provided by client
code.  Default \trucler{} methods do not specialize to this parameter,
but client code can provide primary or auxiliary methods that do
specialize to that parameter, making these methods applicable only for
that client.

\trucler{} provides a \emph{reference} implementation that can be used
by new \commonlisp{} implementations.  But \trucler{} has the
possibility of using client-specific environment with its protocol.
This feature allows the creation of compilers that can be used in
different \commonlisp{} implementations without the need to adapt the
environment-manipulation code to each implementation.  Furthermore,
\trucler{} provides two such \emph{native} implementations, namely for
\sbcl{} and \ccl{}.  We welcome contributions to \trucler{} in the
form of more such native implementations.

