\chapter{RISC-V}
\label{chapter-backend-risc-v}

\section{Register usage}
\label{sec-backend-risc-v-register-usage}

\section{Calling conventions}
\label{sec-backend-risc-v-calling-conventions}

These calling conventions are based on the following ideas:

\begin{enumerate}
\item The calling conventions are used only when a call to an unknown
  function is made.  When the call is to a known, globally defined,
  function, the call-site manager bypasses the argument-parsing code
  and stores the arguments directly where the callee expects them.
  Therefore, the performance of the calling conventions is not
  extremely important.
\item The result of evaluating a form for which all values are needed,
  is that the values are pushed on top of the stack.
\end{enumerate}

\subsection{Normal call}

Note that the RISC-V architecture requires the stack pointer to be
aligned to $16$ bytes.

\begin{enumerate}
\item Compute the callee function and the arguments into a temporary
  locations.
\item Access the slot in the function object containing the initial
  frame size for the callee and put the result in some variable, say
  $v$.  This number is a multiple of $16$.
\item The number of arguments is known statically.  Call that number
  plus $2$ ($1$ for the saved caller frame pointer and $1$ for the
  call-site descriptor) $N$.  

\end{enumerate}

% LocalWords:  callee
