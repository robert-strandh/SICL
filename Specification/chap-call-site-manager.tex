\chapter{Call-site manager}
\label{chap-call-site-manager}

The contents of this chapter is a summary and an update of the paper
we published in ELS 2021 \cite{DBLP:conf/els/Strandh21}.

The default calling conventions are described in
\refChap{chap-calling-conventions}.  In summary:

\begin{itemize}
\item we do not use callee-saves registers,
\item the caller sets up the stack frame for the callee,
\item all arguments are passed on the stack on top of the callee stack
  frame, and
\item return values beyond the first one are returned on the stack.
\end{itemize}

Parsing the arguments is done by a part of the callee function
called the \emph{prelude}.  The result of this action is to initialize
lambda-list parameters stored either in registers or in the stack
frame.  The last action of the prelude is to remove the arguments from
the stack.

If the call site consists of the name of a function followed by forms
that compute the arguments, then the code for the default calling
conventions is not directly present in the caller.  Instead, the call
site consists of a single unconditional \texttt{jump} instruction.
The target of this instruction is called a \emph{trampoline snippet}
or just a \emph{snippet} for short.  The call-site manager allocates a
new snippet when required, and alters the target of the \texttt{jump}
instruction.  For a general call, the snippet contains the
instructions of the default calling conventions, and those are
generated by the call-site manager.  The last instruction of the
snippet is another unconditional \texttt{jump} instruction that jumps
back to the instruction following the first \texttt{jump} instruction.

The snippet is recomputed whenever the callee is updated, such as when
a call to \texttt{(setf fdefinition)} is made.  The snippet is thus
customized for the callee in the following ways:

\begin{enumerate}
\item The static environment is a constant and does not have to be
  loaded from the rack.  If the callee does not close over any
  variables, the static environment does not have to be accessed at
  all.
\item The frame size of the callee is a constant and does not
  have to be loaded from the rack
\item The entry point is a constant and does not have to be loaded
  from the rack.
\end{enumerate}

Already, these constants save many memory operations, thereby making
the call more efficient.  However, the prelude of a function is
followed by a \emph{general body} of the function, and that body
contains instructions for communicating with the debugger as describe
in \refChap{chap-debugger}.  So the call-site manager uses the entry
point of the prelude only when a breakpoint has been set in the
callee.

The compiler generates an optimized function body to be used when
there are no breakpoints.  This optimized function body is not
preceded by a prelude for parsing arguments.  Instead, the call-site
manager uses information provided in the callee to access arguments in
the caller and put them in the locations expected by the optimized
function body.  Thus, there is no argument parsing performed by the
callee, which is particularly advantageous when the callee has
optional and/or keyword parameters and the call-site provides those
arguments in the most common way.%
\footnote{So that keyword arguments are pairs, each one of which is a
  symbol in the \texttt{keyword} package followed by a value.}
In that case, the call-site manager generates code to access the
relevant argument directly and put it directly where the optimize
function body of the callee expects it.

Initial versions of \sysname{} will not include a call-site manager.
Each call site will contain the general call sequence described
above.

% LocalWords:  callee
