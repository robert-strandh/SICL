\chapter{Environments}
\label{chap-environments}

\cl{} has a concept of \emph{environments}, and in fact several
different environments and several different \emph{kinds} of
environment are mentioned in the HyperSpec.  However, \cl{} does not
mandate any particular representation of these environments, nor does
it mention any particular \emph{operations} on environments other than
the implicit operations of defining functions, variables, macros,
types, etc. 

\section{The global environment}
\label{sec-the-global-environment}

In many \cl{} systems the global environment is \emph{spread out} in
that it does not have an explicit definition as a data type.  Parts of
it might be contained in global locations such as the set of packages
or the set of classes.  Other parts of it may be stored in symbols
such as the value or the function definition of a symbol.  The
standard specifically allows for this kind of spread-out
representation.  

In \sysname{}, we prefer to have an explicit representation of the
global environment as a data object.  By doing it this way, we can
allow for any number of global environments present in the system at
any point in time.  Different global environments can have a different
set of packages, a different set of classes, a different set of types,
etc.  This representation can give us several interesting advantages: 

\begin{itemize}
\item We might ensure that there is always a \emph{sane} environment
  present in case some environment gets destroyed (by a user
  accidentally removing some essential system function, for instance).
\item We can allow for several different packages with the same name
  to exist in a system, as long as they are present in different
  environments, which would allow for simpler experimentation with
  different versions of packages. 
\item We could even imagine a multi-user system based on different
  environments, and we could then allow users to do things such as
  defining \texttt{:after} methods on \texttt{print-object} that are
  private to that user. 
\item etc.
\end{itemize}

A global environment in \sysname{} would then contain:

\begin{itemize}
\item A set of \emph{packages}, represented either as a list or as a
  hash table mapping names to packages.
\item A dictionary of \emph{classes}, represented either as an
  association list or as a hash table mapping names to classes.
\item A mapping from function names to functions, macros, and special
  operators.
\item Type information for names of functions.
\item A mapping from names to type definitions.
\item A mapping from names to \emph{variables}.
\item Values of \emph{constant variables}.
\item etc.\fixme{State exactly what the global environment contains.}
\end{itemize}
