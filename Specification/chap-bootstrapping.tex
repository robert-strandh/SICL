\chapter{Bootstrapping}

An image I is said to be \emph{viable} if and only if it is possible
to obtain a complete \cl{} system by starting with I and loading a
sequence of ordinary compiled files.

There are many ways of making a viable image.  We are interested in
making a \emph{small} viable image.  In particular, we want the
initial image \emph{not to contain the compiler}, because we want to
be able to load the compiler as a bunch of compiled files. 

By requiring the viable image not to contain the compiler, we place
some restrictions on the code in it.  

One such restriction is that creating the discriminating function of a
generic function is not allowed to invoke the compiler.  Since doing
so is typical in any CLOS implementation, we must find a different
way.

Another such restriction concerns the creation of classes.  Typically,
when a class is created, the reader and writer methods are created by
invoking the compiler.  Again, a different way must be found.

In order to create an instance of a class, the following functions
are called, so they must exist and be executable:

\begin{itemize}
\item \texttt{make-instance}.  Called directly to create the instance.
\item \texttt{allocate-instance}.  Called by \texttt{make-instance} to
  allocate storage for the instance.
\item \texttt{initialize-instance}.  Called by \texttt{make-instance}
  to initialize the slots of the instance.
\item \texttt{shared-initialize}.  Called by
  \texttt{initialize-instance}.
\item \texttt{slot-value}.  Called by \texttt{shared-initialize}.
\item \texttt{(setf slot-value)}.  Called by
  \texttt{shared-initialize}.
\end{itemize}

Most of these functions are generic functions, so the functionality
that implements generic function dispatch must also be in place,
including:

\begin{itemize}
\item \texttt{compute-discriminating-function}.
\item \texttt{compute-applicable-methods}.
\item \texttt{compute-applicable-methods-using-classes}.
\item \texttt{compute-effective-method}.  This function returns a form
  to be compiled, but since we do not have the compiler, we instead
  use some more direct method similar to what is shown in chapter 1 of
  the AMOP.  Furthermore, in order to take into account different
  method combinations, we also need the compiler, so only the method
  combinations used by the compiler will be considered at
  bootstrapping.
\end{itemize}
