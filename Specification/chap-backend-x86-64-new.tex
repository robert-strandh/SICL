\chapter{x86-64 (new version)}
\label{chapter-backend-x86-64-new}

\section{Register usage}
\label{sec-backend-x86-64-new-register-use}

Most conventions for register usage are based on the idea that many
registers need to permanently assigned to things like arguments and
return values so that function calls between separately compiled
functions can be as efficient as possible.  But one main feature of
\sysname{} is the call-site manager which uses information about both
the caller and the callee to transmit arguments and return values.  As
a result, the default calling conventions are used only in some
cases.  Therefore, it is more important to make the default calling
conventions as simple as possible.

Furthermore, we do not use any callee-saves registers.  The call-site
manager is able to optimize register use for leaf routines, by
essentially treating every register as a callee-saves register, except
that this optimization is more flexible, since, again, the call-site
manager has knowledge about both the caller and the callee.  By not
using callee-saves registers, we are also able to simplify both the
register allocator, the compiler, and the root-finding phase of the
garbage collector.

As a result of these considerations, most registers can be used as
general-purpose registers for holding lexical variables.  The
following table shows the resulting register usage:

\begin{tabular}{|l|l|l|}
\hline
Name & Used for\\
\hline
\hline
RAX & First return value\\
RBX & Dynamic environment\\
RCX & Register variable\\
RDX & Register variable\\
RSP & Stack pointer \\
RBP & Frame pointer \\
RSI & Register variable\\
RDI & Register variable\\
R8  & Register variable\\
R9  & Register variable\\
R10 & Static env. argument\\
R11 & Register variable\\
R12 & Register variable\\
R13 & Register variable\\
R14 & Register variable\\
R15 & Register variable\\
\hline
\end{tabular}

We use the \texttt{FS} segment register to hold an instance of the
\texttt{thread} class.  Such an instance will be used in a variety of
situations including debugging support, etc.

For information about the call-site descriptor, see
\refSec{sec-call-site-descriptor}.


\section{Calling conventions}

These calling conventions are based on two ideas:

\begin{enumerate}
\item The calling conventions are used in very few cases.  Therefore,
  the performance of the calling conventions is not extremely
  important.
\item The result of evaluating a form for which all values are needed,
  is that the values are pushed on top of the stack.
\end{enumerate}

\subsection{Normal call}

\begin{enumerate}
\item Compute the callee function and the arguments into a temporary
  locations.
\item Store RSP in a lexical variable, say $v$.
\item Push RBP
\item Load the rack of the callee function into a temporary register.
\item Load the static environment of the callee from the rack of the
  callee function object into R10.
\item Load the initial frame size of the callee (which
  includes the return address and the call-site descriptor, in
  addition to all lexical variables required to parse arguments) from
  the rack of the callee function object into a temporary register.
\item Subtract from RSP the initial frame size computed in the
  previous step.
\item Push arguments.
\item Load the dynamic environment into RBX.
\item Subtract $v+1$ from RBP, thereby establishing the base pointer
  for the callee.
\item Store the call-site descriptor on the stack in RBP-16.
\item Execute the CALL instruction to the entry point.
\end{enumerate}

The stack frame of the callee upon function entry is illustrated in
\refFig{fig-x86-64-alternative-stack-frame-at-entry}.

\begin{figure}
\begin{center}
\inputfig{fig-x86-64-alternative-stack-frame-at-entry.pdf_t}
\end{center}
\caption{\label{fig-x86-64-alternative-stack-frame-at-entry}
Stack frame at entry.}
\end{figure}

\subsection{Function entry}

Prelude:

\begin{enumerate}
\item Pop the return address off the stack and store it into RBP-8.
\item If required, move the static environment from R10 into some
  other lexical location, and dynamic environment from RBX into some
  other lexical location.
\item Parse the arguments and store the corresponding parameters in
  lexical locations, either in registers or in the initial stack
  frame.
\end{enumerate}

After the execution of the function prelude, the contents of the stack
frame is as illustrated in
\refFig{fig-x86-64-alternative-stack-frame-after-arguments-parsed}.

\begin{figure}
\begin{center}
\inputfig{fig-x86-64-alternative-stack-frame-after-arguments-parsed.pdf_t}
\end{center}
\caption{\label{fig-x86-64-alternative-stack-frame-after-arguments-parsed}
Stack frame after arguments parsed.}
\end{figure}

% LocalWords:  callee
