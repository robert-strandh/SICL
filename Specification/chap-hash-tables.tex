\chapter{Hash tables}

This module was written by Hayley Patton.  The current location for
this module is in the \sysname{} repository in the directory
\texttt{Code/Hash-tables}, but we may extract it to a separate
repository in the future.

\section{Package}

The package for all symbols in this chapter is \texttt{sicl-hash-table}.

\section{Protocol}

Most of the standard functions on hash tables are implemented as
generic functions:

{\small\Defgeneric {hash-table-p} {hash-table}
}

{\small\Defgeneric {hash-table-count} {hash-table}
}

{\small\Defgeneric {hash-table-size} {hash-table}
}

{\small\Defgeneric {hash-table-rehash-size} {hash-table}
}

{\small\Defgeneric {hash-table-rehash-threshold} {hash-table}
}

{\small\Defgeneric {gethash} {key hash-table \optional default}
}

{\small\Defgeneric {(setf gethash)} {new-value key hash-table \optional default}
}

{\small\Defgeneric {hash-table-test} {hash-table}
}

{\small\Defgeneric {remhash} {key hash-table}
}

{\small\Defgeneric {clrhash} {hash-table}
}

{\small\Defgeneric {maphash} {hash-table}
}

Some additional generic functions are provided, which should be implemented
by a hash table implementation:

\Defgeneric {make-hash-table-iterator} {hash-table}

Return a function which implements the iterator of
\cl{with-hash-table-iterator}.

Furthermore, some generic functions will be useful for implementing a hash
table:

\Defgeneric {hash-table-hash-function} {hash-table}

Return a function which accepts an \term{offset value} of type
\cl{(unsigned-byte 64)}, and a key to hash, returning a hash of type
\cl{(unsigned-byte 64)}.

A random offset is generated per hash table, and is used to avoid an
\term{algorithmic complexity attack}, where an adversary could (indirectly)
insert keys that they know will all collide, greatly slowing down an
application. It is expected that this offset will be used to perturb the
hashes generated, perhaps by being used as the initial state of a hashing
algorithm.

\Defgeneric {\%hash-table-test} {hash-table}

Return the test function used for comparing keys. This function is necessary
because \cl{hash-table-test} must return a symbol which designates a
standardized test function, and not the function itself; however, an
implementor of a hash table is likely to want to avoid accessing the global
environment when probing keys.

\section{Base class}

\Defclass {hash-table}

This class is the base class of all hash tables.  It is a subclass of
the class \texttt{standard-object}.

\section{Implementation}

\subsection{Hash table implemented as a list of entries}

\Defclass {list-hash-table}

This class is a subclass of the class \texttt{hash-table}.
It provides and implementation of the protocol where the entries are
stored as an association list where the key is the \texttt{car} of the
element in the list and the value is the \texttt{cdr} of the element
in the list.

{\small\Defmethod {gethash} {key (hash-table {\tt list-hash-table})
    \optional default}
}

This method calls the generic function \texttt{contents} with
\textit{hash-table} as an argument to obtain a list of entries of
\texttt{hash-table}.  It also calls the generic function
\texttt{hash-table-test} with \textit{hash-table} as an argument to
obtain a function to be used to compare the keys of the entries to
\texttt{key}.  It then calls the standard \commonlisp{} function
\texttt{assoc}, passing it \textit{key}, the list of entries, and the
test function as the value of the keyword argument \texttt{:test}.  If
the call returns a non-\texttt{nil} value (i.e. a valid entry), then
the method returns two values, the \texttt{cdr} of that entry and
\texttt{t}.  Otherwise, the method return \texttt{nil} and
\texttt{nil}.

\subsection{Hash table implemented as a vector of buckets}

\Defclass {bucket-hash-table}

This class is a subclass of \texttt{hash-table}.  The implementation
uses a vector of buckets.
