\documentclass{beamer}
\usepackage[latin1]{inputenc}
\beamertemplateshadingbackground{red!10}{blue!10}
%\usepackage{fancybox}
\usepackage{epsfig}
\usepackage{verbatim}
\usepackage{url}
%\usepackage{graphics}
%\usepackage{xcolor}
\usepackage{fancybox}
\usepackage{moreverb}
%\usepackage[all]{xy}
\usepackage{listings}
\usepackage{filecontents}
\usepackage{graphicx}

\lstset{
  language=Lisp,
  basicstyle=\scriptsize\ttfamily,
  keywordstyle={},
  commentstyle={},
  stringstyle={}}

\def\inputfig#1{\input #1}
\def\inputeps#1{\includegraphics{#1}}
\def\inputtex#1{\input #1}

\inputtex{logos.tex}

%\definecolor{ORANGE}{named}{Orange}

\definecolor{GREEN}{rgb}{0,0.8,0}
\definecolor{YELLOW}{rgb}{1,1,0}
\definecolor{ORANGE}{rgb}{1,0.647,0}
\definecolor{PURPLE}{rgb}{0.627,0.126,0.941}
\definecolor{PURPLE}{named}{purple}
\definecolor{PINK}{rgb}{1,0.412,0.706}
\definecolor{WHEAT}{rgb}{1,0.8,0.6}
\definecolor{BLUE}{rgb}{0,0,1}
\definecolor{GRAY}{named}{gray}
\definecolor{CYAN}{named}{cyan}

\newcommand{\orchid}[1]{\textcolor{Orchid}{#1}}
\newcommand{\defun}[1]{\orchid{#1}}

\newcommand{\BROWN}[1]{\textcolor{BROWN}{#1}}
\newcommand{\RED}[1]{\textcolor{red}{#1}}
\newcommand{\YELLOW}[1]{\textcolor{YELLOW}{#1}}
\newcommand{\PINK}[1]{\textcolor{PINK}{#1}}
\newcommand{\WHEAT}[1]{\textcolor{wheat}{#1}}
\newcommand{\GREEN}[1]{\textcolor{GREEN}{#1}}
\newcommand{\PURPLE}[1]{\textcolor{PURPLE}{#1}}
\newcommand{\BLACK}[1]{\textcolor{black}{#1}}
\newcommand{\WHITE}[1]{\textcolor{WHITE}{#1}}
\newcommand{\MAGENTA}[1]{\textcolor{MAGENTA}{#1}}
\newcommand{\ORANGE}[1]{\textcolor{ORANGE}{#1}}
\newcommand{\BLUE}[1]{\textcolor{BLUE}{#1}}
\newcommand{\GRAY}[1]{\textcolor{gray}{#1}}
\newcommand{\CYAN}[1]{\textcolor{cyan }{#1}}

\newcommand{\reference}[2]{\textcolor{PINK}{[#1~#2]}}
%\newcommand{\vect}[1]{\stackrel{\rightarrow}{#1}}

% Use some nice templates
\beamertemplatetransparentcovereddynamic

\newcommand{\A}{{\mathbb A}}
\newcommand{\degr}{\mathrm{deg}}

\title{Fast, Maintainable, and Portable Sequence Functions}

\author{Ir�ne Durand \& Robert Strandh}
\institute{
LaBRI, University of Bordeaux
}
\date{April, 2017}

%\inputtex{macros.tex}


\begin{document}
\frame{
\resizebox{3cm}{!}{\includegraphics{Logobx.pdf}}
\hfill
\resizebox{1.5cm}{!}{\includegraphics{labri-logo.pdf}}
\titlepage
\vfill
\small{European Lisp Symposium, Brussels, Belgium \hfill ELS2017}
}

\setbeamertemplate{footline}{
\vspace{-1em}
\hspace*{1ex}{~} \GRAY{\insertframenumber/\inserttotalframenumber}
}

\frame{
\frametitle{Context: The \sicl{} project}

https://github.com/robert-strandh/SICL

Several objectives:

\begin{itemize}
\item Create high-quality \emph{modules} for implementors of
  \commonlisp{} systems.
\item Improve existing techniques with respect to algorithms and data
  structures where possible.
\item Improve readability and maintainability of code.
\item Improve documentation.
\item Ultimately, create a new implementation based on these modules.
\end{itemize}
}


\frame{
\frametitle{The challenge}

  \begin{itemize}
  \item Good performance requires specialization on element type (for
    vector), test functions, key functions, and values of certain
    parameters such as \texttt{:end} and \texttt{:from-end} (for
    lists).
  \item A separate function for each case makes the source code huge
    and hard to test.
  \item Using macros to generate separate functions makes the code
    hard to understand.
  \end{itemize}
}

\frame{
\frametitle{Previous work}

At ELS 2015 in London, we presented a technique for processing lists
from the end (when \texttt{:from-end} is true).  The present work uses
that technique to obtain one special case.
}

\frame{
\frametitle{Previous work}

\begin{itemize}
\item \ecl{} Uses macros similar to ours, but does not specialize
\item \sbcl{} Uses a mixed approach.  It Requires the caller to specify
  type in order to be fast.
\item The CLISP sequence functions are implemented in C.
\end{itemize}

}

\frame{
\frametitle{Our technique}

\begin{itemize}
\item We use macros for each required specialization.
\item The macros duplicate the body so that the compiler can
  specialize each case.
\end{itemize}
}

\frame[containsverbatim]{
\frametitle{Example: handling \texttt{:end}}

To illustrate our technique, we use a simplified version of
\texttt{find}.

\begin{verbatim}
(defun find (item list end)
  (loop for element in list
        for index from 0
        until (and end (>= index end))
        when (eql item element)
          return element))
\end{verbatim}

Whether \texttt{end} is \texttt{nil} or not is loop invariant, but
obviously not known at compile time, so a typical compiler can not
do anything with this information.
}
\frame[containsverbatim]{
\frametitle{Example: handling \texttt{:end}}

To get better performance, we could do this:

\begin{verbatim}
(defun find (item list end)
  (if end
      (loop for element in list
            for index from 0
            until (and end (>= index end))
            when (eql item element)
              return element)
      (loop for element in list
            when (eql item element)
              return element)))
\end{verbatim}

But now we have code duplication which makes this version less
maintainable and harder to test.
}

\frame[containsverbatim]{
\frametitle{Example: handling \texttt{:end}}

Our technique wraps the loop in a macro call like this:

\begin{verbatim}
(defun find (item list end)
  (with-end end
    (loop for element in list
          for index from 0
          until (and end (>= index end))
          when (eql item element)
            return element)))
\end{verbatim}

}

\frame[containsverbatim]{
\frametitle{Example: handling \texttt{:end}}

The macro is defined like this:

\begin{verbatim}
(defmacro with-end (end-var &body body)
  `(if ,end-var
       (progn ,@body)
       (progn ,@body)))
\end{verbatim}

}


\frame[containsverbatim]{
\frametitle{Example: handling \texttt{:end}}

The expanded \texttt{find} function looks like this:

\begin{verbatim}
(defun find (item list end)
  (if end
      (progn (loop for element in list
                   for index from 0
                   until (and end (>= index end))
                   when (eql item element)
                     return element))
      (progn (loop for element in list
                   for index from 0
                   until (and end (>= index end))
                   when (eql item element)
                     return element))))
\end{verbatim}

}


\frame[containsverbatim]{
\frametitle{Example: handling \texttt{:end}}

A good compiler now notices that \texttt{end} is \texttt{nil} in the
\emph{else} branch of the if, so the \texttt{until} clause can be
removed, and as a consequence, the \texttt{index} variable is unused.
The resulting code becomes:

\begin{verbatim}
(defun find (item list end)
  (if end
      (progn (loop for element in list
                   for index from 0
                   until (and end (>= index end))
                   when (eql item element)
                     return element))
      (progn (loop for element in list
                   when (eql item element)
                     return element))))
\end{verbatim}

}

\frame{
\frametitle{Other specializations}

We handle the following specializations in a similar way:

\begin{itemize}
\item We check for the \texttt{:key} parameter being \texttt{identity}
  or \texttt{car}, so that the call to the \texttt{key} function can
  be eliminated, or inlined.
\item We check for the \texttt{:test} parameter being \texttt{eq} or
  \texttt{eql}, so that the call to the \texttt{test} function can
  be inlined in those cases.
\item We specialize for the element type when the sequence is a
  vector.
\end{itemize}

}

\frame{
\frametitle{Advantages to our technique}

\begin{itemize}
\item The code is much easier to understand, and therefore to
  maintain.
\item Testing is simplified because not all combinations of parameter
  values and element types need to be tested.
\item The resulting code is very fast.
\item The code is independent of the implementation.  Only the
  implementation of the macros are specific to each implementation.
\end{itemize}

}

\frame{
\frametitle{Disadvantages}

\begin{itemize}
\item The resulting code of a single function is very large before the
  compiler can optimize it, resulting in long compilation times, but
  especially requiring huge amounts of memory during compilation.
\item In any particular branch of an \emph{if} form, there is unused
  code, resulting in an avalanche of compiler messages.
\end{itemize}

}

\frame{
\frametitle{Future work}

\begin{itemize}
\item Determine policy.
\end{itemize}

}

\frame{
  \frametitle{Acknowledgments}

We would like to thank Bart Botta, Pascal Bourguignon, and Philipp
Marek for providing valuable feedback on early versions of this paper.
}

\frame{
\frametitle{Thank you}

Questions?
}

%% \frame{\tableofcontents}
%% \bibliography{references}
%% \bibliographystyle{alpha}

\end{document}
