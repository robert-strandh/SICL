\documentclass[format=sigconf]{acmart}
\usepackage[utf8]{inputenc}

\def\inputfig#1{\input #1}
\def\inputtex#1{\input #1}
\def\inputal#1{\input #1}
\def\inputcode#1{\input #1}

\inputtex{logos.tex}
\inputtex{refmacros.tex}
\inputtex{other-macros.tex}

\acmConference[ELS'21]{the 14th European Lisp Symposium}{April 27--28 2021}{%
  Online}
%\acmISBN{978-2-9557474-3-8}
\acmISBN{}
\acmDOI{10.5281/zenodo.3747548}
\startPage{1}
\setcopyright{rightsretained}
\copyrightyear{2020}

\begin{document}
\title{Call-site optimization for Common Lisp}

\author{Robert Strandh}
\email{robert.strandh@u-bordeaux.fr}

\affiliation{
  \institution{LaBRI, University of Bordeaux}
  \streetaddress{351 cours de la libération}
  \city{Talence}
  \country{France}}

\begin{abstract}
Call site optimization. bla bla
\end{abstract}


\begin{CCSXML}
<ccs2012>
<concept>
<concept_id>10011007.10011074.10011099.10011102.10011103</concept_id>
<concept_desc>Software and its engineering~Software testing and debugging</concept_desc>
<concept_significance>500</concept_significance>
</concept>
<concept>
<concept_id>10011007.10011006.10011041.10011048</concept_id>
<concept_desc>Software and its engineering~Runtime environments</concept_desc>
<concept_significance>500</concept_significance>
</concept>
</ccs2012>
\end{CCSXML}

\ccsdesc[500]{Software and its engineering~Software testing and debugging}
\ccsdesc[500]{Software and its engineering~Runtime environments}

\keywords{\clos{}, \commonlisp{}, Compilation, Debugging}

\maketitle


\inputtex{sec-introduction.tex}
\inputtex{sec-previous.tex}
\inputtex{sec-sicl.tex}
\inputtex{sec-our-method.tex}
\inputtex{sec-benefits.tex}
\inputtex{sec-disadvantages.tex}
\inputtex{sec-conclusions.tex}
\inputtex{sec-acknowledgments.tex}

\bibliographystyle{plainnat}
%\bibliographystyle{abbrv}
\bibliography{call-site-optimization}
\end{document}
