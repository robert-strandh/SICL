\documentclass{beamer}
\usepackage[latin1]{inputenc}
\beamertemplateshadingbackground{red!10}{blue!10}
%\usepackage{fancybox}
\usepackage{epsfig}
\usepackage{verbatim}
\usepackage{url}
%\usepackage{graphics}
%\usepackage{xcolor}
\usepackage{fancybox}
\usepackage{moreverb}
%\usepackage[all]{xy}
\usepackage{listings}
\usepackage{filecontents}
\usepackage{graphicx}

\lstset{
  language=Lisp,
  basicstyle=\scriptsize\ttfamily,
  keywordstyle={},
  commentstyle={},
  stringstyle={}}

\def\inputfig#1{\input #1}
\def\inputeps#1{\includegraphics{#1}}
\def\inputtex#1{\input #1}

\inputtex{logos.tex}

%\definecolor{ORANGE}{named}{Orange}

\definecolor{GREEN}{rgb}{0,0.8,0}
\definecolor{YELLOW}{rgb}{1,1,0}
\definecolor{ORANGE}{rgb}{1,0.647,0}
\definecolor{PURPLE}{rgb}{0.627,0.126,0.941}
\definecolor{PURPLE}{named}{purple}
\definecolor{PINK}{rgb}{1,0.412,0.706}
\definecolor{WHEAT}{rgb}{1,0.8,0.6}
\definecolor{BLUE}{rgb}{0,0,1}
\definecolor{GRAY}{named}{gray}
\definecolor{CYAN}{named}{cyan}

\newcommand{\orchid}[1]{\textcolor{Orchid}{#1}}
\newcommand{\defun}[1]{\orchid{#1}}

\newcommand{\BROWN}[1]{\textcolor{BROWN}{#1}}
\newcommand{\RED}[1]{\textcolor{red}{#1}}
\newcommand{\YELLOW}[1]{\textcolor{YELLOW}{#1}}
\newcommand{\PINK}[1]{\textcolor{PINK}{#1}}
\newcommand{\WHEAT}[1]{\textcolor{wheat}{#1}}
\newcommand{\GREEN}[1]{\textcolor{GREEN}{#1}}
\newcommand{\PURPLE}[1]{\textcolor{PURPLE}{#1}}
\newcommand{\BLACK}[1]{\textcolor{black}{#1}}
\newcommand{\WHITE}[1]{\textcolor{WHITE}{#1}}
\newcommand{\MAGENTA}[1]{\textcolor{MAGENTA}{#1}}
\newcommand{\ORANGE}[1]{\textcolor{ORANGE}{#1}}
\newcommand{\BLUE}[1]{\textcolor{BLUE}{#1}}
\newcommand{\GRAY}[1]{\textcolor{gray}{#1}}
\newcommand{\CYAN}[1]{\textcolor{cyan }{#1}}

\newcommand{\reference}[2]{\textcolor{PINK}{[#1~#2]}}
%\newcommand{\vect}[1]{\stackrel{\rightarrow}{#1}}

% Use some nice templates
\beamertemplatetransparentcovereddynamic

\newcommand{\A}{{\mathbb A}}
\newcommand{\degr}{\mathrm{deg}}

\title{Partial Inlining Using Local Graph Rewriting}

\author{Ir�ne Durand \& Robert Strandh}
\institute{
LaBRI, University of Bordeaux
}
\date{April, 2018}

%\inputtex{macros.tex}


\begin{document}
\frame{
\resizebox{3cm}{!}{\includegraphics{Logobx.pdf}}
\hfill
\resizebox{1.5cm}{!}{\includegraphics{labri-logo.pdf}}
\titlepage
\vfill
\small{European Lisp Symposium, Marbella, Spain \hfill ELS2018}
}

\setbeamertemplate{footline}{
\vspace{-1em}
\hspace*{1ex}{~} \GRAY{\insertframenumber/\inserttotalframenumber}
}

\begin{frame}
\frametitle{Context: The \sicl{} project}

https://github.com/robert-strandh/SICL
\vskip 0.5cm
In particular, the \cleavir{} implementation-independent compiler
framework that is currently part of \sicl{}
\end{frame}

\begin{frame}
\frametitle{High-level Intermediate Representation}

\cleavir{} uses (at least) two intermediate representations:
\vskip 0.5cm
\begin{itemize}
\item Abstract Syntax Trees (ASTs) created from source code and a
  global environment.
\item High-level Intermediate Representation (HIR) created from ASTs.
\end{itemize}
\end{frame}

\begin{frame}
\frametitle{The challenge}

\end{frame}

\begin{frame}
\frametitle{Previous work}

Most work focuses on \emph{when} to inline.
\vskip 1cm
\emph{How} to inline is not discussed much, because as Chang and Hwu
put it:  ``The work required to duplicate the callee is trivial''
\end{frame}

\begin{frame}[fragile]
\frametitle{Trivial in functional programming}

The mechanism of inlining is trivial in the context of functional
programming.
\vskip 0.5cm
Simply replace the call by a copy of the body of the called function,
with each occurrence of a parameter replaced by the corresponding
argument.
\vskip 0.5cm

\begin{verbatim}
(defun f (x y) (+ x (* x y)))
\end{verbatim}

\begin{verbatim}
(defun g (a) (f (+ a 2) 234))
\end{verbatim}

becomes

\begin{verbatim}
(defun g (a) (+ (+ a 2) (* (+ a 2) 234)))
\end{verbatim}
\end{frame}

\begin{frame}[fragile]
\frametitle{Not trivial in the presence of side effects}

The mechanism of inlining is not trivial in the context of a language
that allows side effects.
\vskip 0.5cm

\begin{verbatim}
(defun f (x y) (setq x y))

(defun g (a) (f a 3) a)
\end{verbatim}

becomes

\begin{verbatim}
(defun g (a) (setq a 3) a)
\end{verbatim}
\end{frame}


\begin{frame}
\frametitle{Future work}
\end{frame}

\begin{frame}
  \frametitle{Acknowledgments}

We would like to thank Bart Botta, Jan Moringen, John Mercouris, and
Alastair Bridgewater for providing valuable feedback on early versions
of this paper.
\end{frame}

\begin{frame}
\frametitle{Thank you}

Questions?
\end{frame}

\end{document}
