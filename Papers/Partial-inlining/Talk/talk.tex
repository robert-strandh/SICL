\documentclass{beamer}
\usepackage[latin1]{inputenc}
\beamertemplateshadingbackground{red!10}{blue!10}
%\usepackage{fancybox}
\usepackage{epsfig}
\usepackage{verbatim}
\usepackage{url}
%\usepackage{graphics}
%\usepackage{xcolor}
\usepackage{fancybox}
\usepackage{moreverb}
%\usepackage[all]{xy}
\usepackage{listings}
\usepackage{filecontents}
\usepackage{graphicx}

\lstset{
  language=Lisp,
  basicstyle=\scriptsize\ttfamily,
  keywordstyle={},
  commentstyle={},
  stringstyle={}}

\def\inputfig#1{\input #1}
\def\inputeps#1{\includegraphics{#1}}
\def\inputtex#1{\input #1}

\inputtex{logos.tex}

%\definecolor{ORANGE}{named}{Orange}

\definecolor{GREEN}{rgb}{0,0.8,0}
\definecolor{YELLOW}{rgb}{1,1,0}
\definecolor{ORANGE}{rgb}{1,0.647,0}
\definecolor{PURPLE}{rgb}{0.627,0.126,0.941}
\definecolor{PURPLE}{named}{purple}
\definecolor{PINK}{rgb}{1,0.412,0.706}
\definecolor{WHEAT}{rgb}{1,0.8,0.6}
\definecolor{BLUE}{rgb}{0,0,1}
\definecolor{GRAY}{named}{gray}
\definecolor{CYAN}{named}{cyan}

\newcommand{\orchid}[1]{\textcolor{Orchid}{#1}}
\newcommand{\defun}[1]{\orchid{#1}}

\newcommand{\BROWN}[1]{\textcolor{BROWN}{#1}}
\newcommand{\RED}[1]{\textcolor{red}{#1}}
\newcommand{\YELLOW}[1]{\textcolor{YELLOW}{#1}}
\newcommand{\PINK}[1]{\textcolor{PINK}{#1}}
\newcommand{\WHEAT}[1]{\textcolor{wheat}{#1}}
\newcommand{\GREEN}[1]{\textcolor{GREEN}{#1}}
\newcommand{\PURPLE}[1]{\textcolor{PURPLE}{#1}}
\newcommand{\BLACK}[1]{\textcolor{black}{#1}}
\newcommand{\WHITE}[1]{\textcolor{WHITE}{#1}}
\newcommand{\MAGENTA}[1]{\textcolor{MAGENTA}{#1}}
\newcommand{\ORANGE}[1]{\textcolor{ORANGE}{#1}}
\newcommand{\BLUE}[1]{\textcolor{BLUE}{#1}}
\newcommand{\GRAY}[1]{\textcolor{gray}{#1}}
\newcommand{\CYAN}[1]{\textcolor{cyan }{#1}}

\newcommand{\reference}[2]{\textcolor{PINK}{[#1~#2]}}
%\newcommand{\vect}[1]{\stackrel{\rightarrow}{#1}}

% Use some nice templates
\beamertemplatetransparentcovereddynamic

\newcommand{\A}{{\mathbb A}}
\newcommand{\degr}{\mathrm{deg}}

\title{Partial Inlining Using Local Graph Rewriting}

\author{Ir�ne Durand \& Robert Strandh}
\institute{
LaBRI, University of Bordeaux
}
\date{April, 2018}

%\inputtex{macros.tex}


\begin{document}
\frame{
\resizebox{3cm}{!}{\includegraphics{Logobx.pdf}}
\hfill
\resizebox{1.5cm}{!}{\includegraphics{labri-logo.pdf}}
\titlepage
\vfill
\small{European Lisp Symposium, Marbella, Spain \hfill ELS2018}
}

\setbeamertemplate{footline}{
\vspace{-1em}
\hspace*{1ex}{~} \GRAY{\insertframenumber/\inserttotalframenumber}
}

\begin{frame}
\frametitle{Context: The \sicl{} project}

https://github.com/robert-strandh/SICL
\vskip 0.5cm
In particular, the \cleavir{} implementation-independent compiler
framework that is currently part of \sicl{}.
\end{frame}

\begin{frame}
\frametitle{High-level Intermediate Representation}

\cleavir{} uses (at least) two intermediate representations:
\vskip 0.5cm
\begin{itemize}
\item Abstract Syntax Trees (ASTs) created from source code and a
  global environment.
\item High-level Intermediate Representation (HIR) created from ASTs.
\end{itemize}
\end{frame}

\begin{frame}
\frametitle{High-level Intermediate Representation}


HIR is similar to the kind of flow graphs used in traditional
compiler design.
\vskip 0.5cm
Main difference: In HIR, only \commonlisp{} objects are manipulated.
\vskip 0.5cm
By restricting HIR data this way, we can apply most of our
optimization techniques to this representation, including type
inference.
\end{frame}

\begin{frame}
\frametitle{HIR instruction categories}
The following categories exist:

\begin{itemize}
\item Low-level accessors such as \texttt{car}, \texttt{cdr},
  \texttt{rplaca}, \texttt{rplacd}, \texttt{aref}, \texttt{aset},
  \texttt{slot-read}, and \texttt{slot-write}.
\item Instructions for low-level arithmetic on, and comparison of,
  floating-point numbers and fixnums.
\item Instructions for testing the type of an object.
\item Instructions such as \texttt{funcall}, \texttt{return}, and
  \texttt{unwind} for handling function calls and returns.
\end{itemize}
\end{frame}

\begin{frame}
  \frametitle{Two particular HIR instructions}

Two HIR instruction types have no correspondence in \commonlisp{}
source code:
\begin{itemize}
\item The \texttt{enter} instruction is the first instruction of a
  sub-graph corresponding to a function.
\item The \texttt{enclose} instruction creates a callable function
  from an \texttt{enter} instruction and the current environment.
\end{itemize}
\end{frame}

\begin{frame}
\frametitle{Previous work}

Most work focuses on \emph{when} to inline.
\vskip 1cm
\emph{How} to inline is not discussed much, because as Chang and Hwu
put it:  ``The work required to duplicate the callee is trivial''
\end{frame}

\begin{frame}[fragile]
\frametitle{Trivial in functional programming}

The mechanism of inlining is trivial in the context of functional
programming.
\vskip 0.5cm
Simply replace the call by a copy of the body of the callee,
with each occurrence of a parameter replaced by the corresponding
argument ($\beta$-reduction).
\vskip 0.5cm

\begin{verbatim}
(defun f (x y) (+ x (* x y)))
\end{verbatim}

\begin{verbatim}
(defun g (a) (f (+ a 2) 234))
\end{verbatim}

becomes

\begin{verbatim}
(defun g (a) (+ (+ a 2) (* (+ a 2) 234)))
\end{verbatim}
\end{frame}

\begin{frame}[fragile]
\frametitle{Not trivial in the presence of side effects}

The mechanism of inlining is not trivial in the context of a language
that allows side effects.  We can not use simple $\beta$-reduction.
\vskip 0.5cm

\begin{verbatim}
(defun f (x y) (setq x y))

(defun g (a) (f a 3) a)
\end{verbatim}

becomes

\begin{verbatim}
(defun g (a) (setq a 3) a)
\end{verbatim}
\end{frame}

\begin{frame}
\frametitle{Our technique: local graph rewriting}
Basic idea:
\begin{figure}
\begin{center}
\inputfig{fig-basic-idea.pdf_t}
\end{center}
\end{figure}
\end{frame}

\begin{frame}
\frametitle{Our technique: restrictions}
Only avoiding call/return is no longer important.
\vskip 0.5cm
We also want to allocate the environment of the callee in the caller.
\vskip 0.5cm
This restriction excludes some situations:
\vskip 0.5cm
\begin{itemize}
\item Some cases when the environment of the callee is captured.
\item When the callee is directly or indirectly recursive.
\end{itemize}
\vskip 0.5cm
We have yet to work out necessary and sufficient conditions.
\end{frame}

\begin{frame}
\frametitle{Running example}
\begin{figure}
\begin{center}
\inputfig{fig41.pdf_t}
\end{center}
\end{figure}
\end{frame}

\begin{frame}
\frametitle{Our technique: worklist}
We maintain a worklist containing:
\vskip 0.5cm
\begin{itemize}
\item A \texttt{funcall} instruction (caller).
\item An \texttt{enter} instruction (callee).
\item The successor instruction of the \texttt{enter} instruction,
  called the \emph{target instruction}.
\item A mapping from lexical variables in the callee that
  have already been duplicated in the caller.
\end{itemize}
\end{frame}

\begin{frame}
\frametitle{Our technique: global information}
We also maintain the following global information:
\vskip 0.5cm
\begin{itemize}
\item A mapping from instructions in the callee that have
  already been inlined, to the corresponding instructions in the
  caller.
\item Information about the ownership of lexical variables referred to
  by the callee.
\end{itemize}
\end{frame}

\begin{frame}
\frametitle{Our technique: initialization}
\begin{itemize}
\item Create a copy of the initial callee environment in the caller.
\item Create an initial worklist containing:
\begin{itemize}
\item The \texttt{funcall} instruction representing the call that
  should be inlined.
\item A \emph{private copy} of the initial \texttt{enter} instruction
  of the function to inline.
\item The successor instruction of the initial \texttt{enter}
  instruction, which is the initial target.
\item The initial lexical variable mapping.
\end{itemize}
\end{itemize}
\end{frame}

\begin{frame}
\frametitle{Initial instruction graph}
\begin{figure}
\begin{center}
\inputfig{fig41.pdf_t}
\end{center}
\end{figure}
\end{frame}

\begin{frame}
\frametitle{Instruction graph after initialization}
\begin{figure}
\begin{center}
\inputfig{fig42.pdf_t}
\end{center}
\end{figure}
\end{frame}

\begin{frame}
\frametitle{Our technique: one of four rules}
In each iteration of our technique, one of the following rules is
applied:
\begin{enumerate}
\item If the target instruction has already been inlined, then use the
  existing inlined copy.  No new worklist item is created.
\item If the target instruction is a \texttt{return} instruction, then
  remove call and fix up.  No new worklist item is created.
\item If the target instruction has a single successor, then inline
  it, mapping lexical variables.  Create one new worklist item.
\item If the target instruction has two successors, then inline
  it, mapping lexical variables.  Also replicate the \texttt{funcall}
  and \texttt{enter} instructions.  Create two new worklist items.
\end{enumerate}
\end{frame}


\begin{frame}
\frametitle{Instruction graph after initialization}
\begin{figure}
\begin{center}
\inputfig{fig42.pdf_t}
\end{center}
\end{figure}
\end{frame}

\begin{frame}
\frametitle{Instruction graph after one inlining step}
\begin{figure}
\begin{center}
\inputfig{fig43.pdf_t}
\end{center}
\end{figure}
\end{frame}

\begin{frame}
\frametitle{Instruction graph after two inlining steps}
\begin{figure}
\begin{center}
\inputfig{fig44.pdf_t}
\end{center}
\end{figure}
\end{frame}

\begin{frame}
\frametitle{Instruction graph after three inlining steps}
\begin{figure}
\begin{center}
\inputfig{fig45.pdf_t}
\end{center}
\end{figure}
\end{frame}

\begin{frame}
\frametitle{Instruction graph after four inlining steps}
\begin{figure}
\begin{center}
\inputfig{fig46.pdf_t}
\end{center}
\end{figure}
\end{frame}

\begin{frame}
\frametitle{Final instruction graph}
\begin{figure}
\begin{center}
\inputfig{fig47.pdf_t}
\end{center}
\end{figure}
\end{frame}

\begin{frame}
\frametitle{Our technique: characteristics}
\begin{itemize}
\item Each iteration preserves the overall semantics.
\item Inlining can be stopped at any point, making it partial.
\item We prove termination even in the presence of loops.
\end{itemize}
\end{frame}

\begin{frame}
\frametitle{Future work}
\begin{itemize}
\item Determine necessary and sufficient conditions for our technique
  to be valid.
\item Investigate consequences of multiple entry points for other
  optimization techniques and analyses.
\end{itemize}
\end{frame}

\begin{frame}
  \frametitle{Acknowledgments}

We would like to thank Bart Botta, Jan Moringen, John Mercouris, and
Alastair Bridgewater for providing valuable feedback on early versions
of this paper.
\end{frame}

\begin{frame}
\frametitle{Thank you}

Questions?
\end{frame}

\end{document}
