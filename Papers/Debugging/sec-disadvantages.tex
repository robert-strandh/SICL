\section{Disadvantages of our technique}

Perhaps the most obvious disadvantage of our technique is that the
size of the code will more than double.  The debugging version of the
function body must implement the same functionality as the
non-debugging version, but in addition to that functionality, it must
also contain code for communication with the debugger.  Furthermore,
since fewer optimizations are applied to the non-debugging version,
even without the code for communication, the debugging version would
be larger than the non-debugging version.

While the additional code will impact the memory footprint of the
system, we do not think it will have any negative influence on
caching.  The two versions of the body are kept separate, and the same
versions is typically executed repeatedly.

A minor disadvantage is the initial test that every function must
perform, plus the initialization of the debug-flag register for every
function call, in both versions of the function body.  These
Because the same body version is executed repeatedly, the branch
prediction logic would very likely make the correct prediction for the
initial test.  And with significant inlining of functions, the number
of function calls is reduced, and therefore also the number of times
the debug-flag register needs to be initialized.

Finally, the fact that the technique proposed in this paper is
incompatible with the way most \commonlisp{} systems work, makes it
unlikely that existing systems will be able to use it.  We are
convinced, however, that our technique will represent a major
advantage in terms of productivity for the application programmer.
