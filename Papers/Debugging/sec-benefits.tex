\section{Benefits of our technique}
\label{sec-benefits}

Our technique differs both from the tradition of debugging in
\unix{}-type systems and from the tradition used in FLOSS
\commonlisp{} systems.

\subsection{Difference compared to \unix{}-like systems}

Whereas \unix{}-like systems typically run the debugger in a different
\emph{process} from that (or those) of the application, with our
technique we run both the debugger and the application in the same
process.

The main advantage of this organization is that communication between
the debugger and the application is greatly simplified.  There is no
need for a wire protocol to encode and decode data in the form of
sequences of octets, simply because with a single process, the address
space is shared between the debugger and the application.  Instead, we
can send data in the form of arbitrarily complex data structures
between the two.

\subsection{Difference compared to most FLOSS \commonlisp{} systems}

Most FLOSS \commonlisp{} implementations have a history that started
before multi-threading was common.  As a result, features such as
breakpoints and tracing are often implemented as modifications to
the code.

For example, in \sbcl{}, the user can choose to
\emph{trace} a function in two different ways.  One way is by means of
\emph{encapsulation}, meaning that the function is not modified, and
instead wrapped in a small routine that then replaces the function as
associated with the function \emph{name}.  The function being traced
is not modified.  The other way is by means of a breakpoint, meaning
that the code of the function being traced \emph{is} modified.

However, in both cases, every caller of the function being traced is
affected, barring a caller that is in possession of the function
object itself, rather than its name.  As a result, it is very likely
impractical to trace system functions that may be used internally by
the system.  For example, tracing \texttt{find} or \texttt{position}
(if at all possible) is likely to generate so much information from
callers that are irrelevant to the user as to make
the information impossible to exploit.  And tracing functions such as
\texttt{print}, \texttt{format}, or \texttt{write} would be entirely
impossible, since the trace output would very likely call these
functions in order to generate the output information meant for the
user.

With our suggested technique, tracing a function does not create an
encapsulation and does not modify the code of the function.  Instead,
the existing code communicates with the debugger, and the debugger,
running in a different thread, is in charge of displaying information
to the user.  As a direct consequence, there are no
restrictions such as those indicated above.  The only possible
restriction has to do with inlining, though it may very well turn out
to be possible to propagate debugging information with inlining,
thereby making it possible to trace, or to set breakpoints in any
function such as \texttt{car} or \texttt{+}.  However, it may turn out
that the inclusion of debugging code in such low-level functions would be
prohibitive in terms of performance of code run under the control of
the debugger.

Finally, a significant advantage to our technique is that the
application programmer does not have to choose between compiling the
code for debugging or for performance.  In most existing systems, in
order for it to be possible to benefit from all the debugging
information possible, the programmer has to compile the code with a
combination of values of the existing \texttt{debug} qualities that is
not optimal for performance.  This limitation means that it is often
necessary to recompile the application for one of the two purposes.
With the technique presented in this paper, no such choice is
required, since both versions of the application are always
available.
