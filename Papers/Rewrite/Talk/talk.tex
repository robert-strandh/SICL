\documentclass{beamer}
\usepackage[latin1]{inputenc}
\beamertemplateshadingbackground{red!10}{blue!10}
%\usepackage{fancybox}
\usepackage{epsfig}
\usepackage{verbatim}
\usepackage{url}
%\usepackage{graphics}
%\usepackage{xcolor}
\usepackage{fancybox}
\usepackage{moreverb}
%\usepackage[all]{xy}
\usepackage{listings}
\usepackage{filecontents}
\usepackage{graphicx}

\lstset{
  language=Lisp,
  basicstyle=\scriptsize\ttfamily,
  keywordstyle={},
  commentstyle={},
  stringstyle={}}

\def\inputfig#1{\input #1}
\def\inputeps#1{\includegraphics{#1}}
\def\inputtex#1{\input #1}

\inputtex{logos.tex}

%\definecolor{ORANGE}{named}{Orange}

\definecolor{GREEN}{rgb}{0,0.8,0}
\definecolor{YELLOW}{rgb}{1,1,0}
\definecolor{ORANGE}{rgb}{1,0.647,0}
\definecolor{PURPLE}{rgb}{0.627,0.126,0.941}
\definecolor{PURPLE}{named}{purple}
\definecolor{PINK}{rgb}{1,0.412,0.706}
\definecolor{WHEAT}{rgb}{1,0.8,0.6}
\definecolor{BLUE}{rgb}{0,0,1}
\definecolor{GRAY}{named}{gray}
\definecolor{CYAN}{named}{cyan}

\newcommand{\orchid}[1]{\textcolor{Orchid}{#1}}
\newcommand{\defun}[1]{\orchid{#1}}

\newcommand{\BROWN}[1]{\textcolor{BROWN}{#1}}
\newcommand{\RED}[1]{\textcolor{red}{#1}}
\newcommand{\YELLOW}[1]{\textcolor{YELLOW}{#1}}
\newcommand{\PINK}[1]{\textcolor{PINK}{#1}}
\newcommand{\WHEAT}[1]{\textcolor{wheat}{#1}}
\newcommand{\GREEN}[1]{\textcolor{GREEN}{#1}}
\newcommand{\PURPLE}[1]{\textcolor{PURPLE}{#1}}
\newcommand{\BLACK}[1]{\textcolor{black}{#1}}
\newcommand{\WHITE}[1]{\textcolor{WHITE}{#1}}
\newcommand{\MAGENTA}[1]{\textcolor{MAGENTA}{#1}}
\newcommand{\ORANGE}[1]{\textcolor{ORANGE}{#1}}
\newcommand{\BLUE}[1]{\textcolor{BLUE}{#1}}
\newcommand{\GRAY}[1]{\textcolor{gray}{#1}}
\newcommand{\CYAN}[1]{\textcolor{cyan }{#1}}

\newcommand{\reference}[2]{\textcolor{PINK}{[#1~#2]}}
%\newcommand{\vect}[1]{\stackrel{\rightarrow}{#1}}

% Use some nice templates
\beamertemplatetransparentcovereddynamic

\newcommand{\A}{{\mathbb A}}
\newcommand{\degr}{\mathrm{deg}}

\title{Removing redundant tests by\\replicating control paths}

\author{Ir�ne Durand \& Robert Strandh}
\institute{
LaBRI, University of Bordeaux
}
\date{April, 2017}

%\inputtex{macros.tex}


\begin{document}
\frame{
\resizebox{3cm}{!}{\includegraphics{Logobx.pdf}}
\hfill
\resizebox{1.5cm}{!}{\includegraphics{labri-logo.pdf}}
\titlepage
\vfill
\small{European Lisp Symposium, Brussels, Belgium \hfill ELS2017}
}

\setbeamertemplate{footline}{
\vspace{-1em}
\hspace*{1ex}{~} \GRAY{\insertframenumber/\inserttotalframenumber}
}

\frame{
\frametitle{Context: The \sicl{} project}

https://github.com/robert-strandh/SICL

Several objectives:

\begin{itemize}
\item Create high-quality \emph{modules} for implementors of
  \commonlisp{} systems.
\item Improve existing techniques with respect to algorithms and data
  structures where possible.
\item Improve readability and maintainability of code.
\item Improve documentation.
\item Ultimately, create a new implementation based on these modules.
\end{itemize}
}

\frame{
\frametitle{Compiler framework}

bla bla
}

\frame{
\frametitle{Previous work}

bla bla

}

\frame[containsverbatim]{
\frametitle{An example}

\begin{verbatim}
  (defun car (x)
    (cond ((consp x) (cons-car x))
          ((null x) nil)
          (t (error 'type-error ...))))

  (defun cdr (x)
    (cond ((consp x) (cons-cdr x))
          ((null x) nil)
          (t (error 'type-error ...))))
\end{verbatim}

}

\frame[containsverbatim]{
\frametitle{An example}

\begin{verbatim}
  (let ((a (car x))
        (b (some-function)
        (c (cdr x)))
    ...)
\end{verbatim}

}

\frame[containsverbatim]{
\frametitle{An example}

\begin{verbatim}
  (let ((a (cond ((consp x) (cons-car x))
                 ((null x) nil)
                 (t (error 'type-error ...)))
        (b (some-function)
        (c (cond ((consp x) (cons-cdr x))
                 ((null x) nil)
                 (t (error 'type-error ...)))
    ...)
\end{verbatim}

}

\frame{\frametitle{Rewrite rules}

\begin{enumerate}
\item If $s$ has no predecessors, then remove it from $S$.
\item If $s$ has an incoming arc labeled \emph{true}, then change the
  head of that arc so that it refers to the successor of $s$ referred
  to by the outgoing arc of $s$ labeled \emph{true}.
\item If $s$ has an incoming arc labeled \emph{false}, then change the
  head of that arc so that it refers to the successor of $s$ referred
  to by the outgoing arc of $s$ labeled \emph{false}.
\item If $s$ has $n>1$ predecessors, then replicate $s$ $n$ times;
  once for each predecessor.  Every replica is inserted into $S$.
  Labels of outgoing control arcs are preserved in the replicas.
\item Let $p$ be the (unique) predecessor of $s$.  Remove $p$ as a
  predecessor of $s$ so that existing immediate predecessors of $p$
  instead become immediate predecessors of $s$.  Insert a replica of
  $p$ in each outgoing control arc of $s$, preserving the label of
  each arc.
\end{enumerate}
}

\frame{\frametitle{Initial instruction graph}

\begin{figure}
\begin{center}
\inputfig{fig-rewrite-1.pdf_t}
\end{center}
\end{figure}
}


\frame{\frametitle{Result after one rewrite}
\begin{figure}
\begin{center}
\inputfig{fig-rewrite-one-and-a-half.pdf_t}
\end{center}
\end{figure}
}

\frame{\frametitle{Result after two rewrites}
\begin{figure}
\begin{center}
\inputfig{fig-rewrite-2.pdf_t}
\end{center}
\end{figure}
}

\frame{\frametitle{Result after replicating the test}
\begin{figure}
\begin{center}
\inputfig{fig-rewrite-3.pdf_t}
\end{center}
\end{figure}
}

\frame{\frametitle{Result after replicating \texttt{setq}}
\begin{figure}
\begin{center}
\inputfig{fig-rewrite-4.pdf_t}
\end{center}
\end{figure}
}

\frame{\frametitle{Result after replicating \texttt{cons-car}}
\begin{figure}
\begin{center}
\inputfig{fig-rewrite-5.pdf_t}
\end{center}
\end{figure}
}

\frame{\frametitle{Result after short-circuit \texttt{consp}}
\begin{figure}
\begin{center}
\inputfig{fig-rewrite-6.pdf_t}
\end{center}
\end{figure}
}

\frame{\frametitle{Result after removing unreachable instructions}
\begin{figure}
\begin{center}
\inputfig{fig-rewrite-7.pdf_t}
\end{center}
\end{figure}
}

\frame{\frametitle{Final result}
\begin{figure}
\begin{center}
\inputfig{fig-rewrite-8.pdf_t}
\end{center}
\end{figure}
}

\frame{
\frametitle{Future work}

\begin{itemize}
\item Determine policy.
\end{itemize}

}

\frame{
  \frametitle{Acknowledgments}

We would like to thank Philipp Marek for providing valuable feedback
on early versions of this paper.
}

\frame{
\frametitle{Thank you}

Questions?
}

%% \frame{\tableofcontents}
%% \bibliography{references}
%% \bibliographystyle{alpha}

\end{document}
