\documentclass{beamer}
\usepackage[latin1]{inputenc}
\beamertemplateshadingbackground{red!10}{blue!10}
%\usepackage{fancybox}
\usepackage{epsfig}
\usepackage{verbatim}
\usepackage{url}
%\usepackage{graphics}
%\usepackage{xcolor}
\usepackage{fancybox}
\usepackage{moreverb}
%\usepackage[all]{xy}
\usepackage{listings}
\usepackage{filecontents}
\usepackage{graphicx}

\lstset{
  language=Lisp,
  basicstyle=\scriptsize\ttfamily,
  keywordstyle={},
  commentstyle={},
  stringstyle={}}

\def\inputfig#1{\input #1}
\def\inputeps#1{\includegraphics{#1}}
\def\inputtex#1{\input #1}

\inputtex{logos.tex}

%\definecolor{ORANGE}{named}{Orange}

\definecolor{GREEN}{rgb}{0,0.8,0}
\definecolor{YELLOW}{rgb}{1,1,0}
\definecolor{ORANGE}{rgb}{1,0.647,0}
\definecolor{PURPLE}{rgb}{0.627,0.126,0.941}
\definecolor{PURPLE}{named}{purple}
\definecolor{PINK}{rgb}{1,0.412,0.706}
\definecolor{WHEAT}{rgb}{1,0.8,0.6}
\definecolor{BLUE}{rgb}{0,0,1}
\definecolor{GRAY}{named}{gray}
\definecolor{CYAN}{named}{cyan}

\newcommand{\orchid}[1]{\textcolor{Orchid}{#1}}
\newcommand{\defun}[1]{\orchid{#1}}

\newcommand{\BROWN}[1]{\textcolor{BROWN}{#1}}
\newcommand{\RED}[1]{\textcolor{red}{#1}}
\newcommand{\YELLOW}[1]{\textcolor{YELLOW}{#1}}
\newcommand{\PINK}[1]{\textcolor{PINK}{#1}}
\newcommand{\WHEAT}[1]{\textcolor{wheat}{#1}}
\newcommand{\GREEN}[1]{\textcolor{GREEN}{#1}}
\newcommand{\PURPLE}[1]{\textcolor{PURPLE}{#1}}
\newcommand{\BLACK}[1]{\textcolor{black}{#1}}
\newcommand{\WHITE}[1]{\textcolor{WHITE}{#1}}
\newcommand{\MAGENTA}[1]{\textcolor{MAGENTA}{#1}}
\newcommand{\ORANGE}[1]{\textcolor{ORANGE}{#1}}
\newcommand{\BLUE}[1]{\textcolor{BLUE}{#1}}
\newcommand{\GRAY}[1]{\textcolor{gray}{#1}}
\newcommand{\CYAN}[1]{\textcolor{cyan }{#1}}

\newcommand{\reference}[2]{\textcolor{PINK}{[#1~#2]}}
%\newcommand{\vect}[1]{\stackrel{\rightarrow}{#1}}

% Use some nice templates
\beamertemplatetransparentcovereddynamic

\newcommand{\A}{{\mathbb A}}
\newcommand{\degr}{\mathrm{deg}}

\title{Removing redundant tests by\\replicating control paths}

\author{Ir�ne Durand \& Robert Strandh}
\institute{
LaBRI, University of Bordeaux
}
\date{April, 2017}

%\inputtex{macros.tex}


\begin{document}
\frame{
\resizebox{3cm}{!}{\includegraphics{Logobx.pdf}}
\hfill
\resizebox{1.5cm}{!}{\includegraphics{labri-logo.pdf}}
\titlepage
\vfill
\small{European Lisp Symposium, Brussels, Belgium \hfill ELS2017}
}

\setbeamertemplate{footline}{
\vspace{-1em}
\hspace*{1ex}{~} \GRAY{\insertframenumber/\inserttotalframenumber}
}

\frame{
\frametitle{Context: The \sicl{} project}

https://github.com/robert-strandh/SICL

Several objectives:

\begin{itemize}
\item Create high-quality \emph{modules} for implementors of
  \commonlisp{} systems.
\item Improve existing techniques with respect to algorithms and data
  structures where possible.
\item Improve readability and maintainability of code.
\item Improve documentation.
\item Ultimately, create a new implementation based on these modules.
\end{itemize}
}

\frame{
\frametitle{Compiler framework}

A large part of \sicl{} is a framework for creating \commonlisp{}
compilers called \cleavir{}.

\vspace{0.5cm}

It is written so that an implementation can adapt it to its needs,
while still benefiting from reasonable default behavior.

\vspace{0.5cm}

In particular, \cleavir{} will contain a large number of standard
compiler optimization algorithms, and a few or our own.
}

\frame{\frametitle{Purpose of current work}

When possible, avoid redundant tests.

\vspace{0.5cm}

A test is redundant if a preceding identical test that dominates this
one exists.

\vspace{0.5cm}

We detect redundant tests in \emph{intermediate code}, and we
eliminate them using \emph{local graph rewriting}.  }

\frame{\frametitle{Previous work}

Frank Mueller and David Whalley presented a paper at PLDI in 1995,
entitled ``Avoiding Conditional Branches by Code Replication''.  

\vspace{0.5cm}

That paper uses ad-hoc techniques to detect redundant tests and
transform the code to avoid them.

\vspace{0.5cm}

We are unaware of any other work in this domain.
}


\frame[containsverbatim]{
\frametitle{An example}

A client implementation might define \texttt{car} and \texttt{cdr}
like this (Clasp and \sicl{} both do):

\begin{verbatim}
  (defun car (x)
    (cond ((consp x) (cons-car x))
          ((null x) nil)
          (t (error 'type-error ...))))

  (defun cdr (x)
    (cond ((consp x) (cons-cdr x))
          ((null x) nil)
          (t (error 'type-error ...))))
\end{verbatim}

Where \texttt{cons-car} and \texttt{cons-car} are primitive operations
that require the argument to be of type \texttt{cons}.

}

\frame[containsverbatim]{
\frametitle{An example}

Now suppose we have the following code to compile:

\begin{verbatim}
  (let ((a (car x))
        (b (some-function)
        (c (cdr x)))
    ...)
\end{verbatim}

}

\frame[containsverbatim]{
\frametitle{An example}

After inlining \texttt{car} and \texttt{cdr}, we get the following code:

\begin{verbatim}
  (let ((a (cond ((consp x) (cons-car x))
                 ((null x) nil)
                 (t (error 'type-error ...)))
        (b (some-function)
        (c (cond ((consp x) (cons-cdr x))
                 ((null x) nil)
                 (t (error 'type-error ...)))
    ...)
\end{verbatim}

}

\frame{\frametitle{Resulting intermediate code}

\begin{figure}
\begin{center}
\inputfig{fig-example-naive.pdf_t}
\end{center}
\end{figure}
}

\frame{\frametitle{Rewrite rules}

\begin{enumerate}
\item If $s$ has no predecessors, then remove it from $S$.
\item If $s$ has an incoming arc labeled \emph{true}, then change the
  head of that arc so that it refers to the successor of $s$ referred
  to by the outgoing arc of $s$ labeled \emph{true}.
\item If $s$ has an incoming arc labeled \emph{false}, then change the
  head of that arc so that it refers to the successor of $s$ referred
  to by the outgoing arc of $s$ labeled \emph{false}.
\item If $s$ has $n>1$ predecessors, then replicate $s$ $n$ times;
  once for each predecessor.  Every replica is inserted into $S$.
  Labels of outgoing control arcs are preserved in the replicas.
\item Let $p$ be the (unique) predecessor of $s$.  Remove $p$ as a
  predecessor of $s$ so that existing immediate predecessors of $p$
  instead become immediate predecessors of $s$.  Insert a replica of
  $p$ in each outgoing control arc of $s$, preserving the label of
  each arc.
\end{enumerate}
}

\frame{\frametitle{Initial instruction graph}

\begin{figure}
\begin{center}
\inputfig{fig-rewrite-1.pdf_t}
\end{center}
\end{figure}
}


\frame{\frametitle{Result after one rewrite}
\begin{figure}
\begin{center}
\inputfig{fig-rewrite-one-and-a-half.pdf_t}
\end{center}
\end{figure}
}

\frame{\frametitle{Result after two rewrites}
\begin{figure}
\begin{center}
\inputfig{fig-rewrite-2.pdf_t}
\end{center}
\end{figure}
}

\frame{\frametitle{Result after replicating the test}
\begin{figure}
\begin{center}
\inputfig{fig-rewrite-3.pdf_t}
\end{center}
\end{figure}
}

\frame{\frametitle{Result after replicating \texttt{setq}}
\begin{figure}
\begin{center}
\inputfig{fig-rewrite-4.pdf_t}
\end{center}
\end{figure}
}

\frame{\frametitle{Result after replicating \texttt{cons-car}}
\begin{figure}
\begin{center}
\inputfig{fig-rewrite-5.pdf_t}
\end{center}
\end{figure}
}

\frame{\frametitle{Result after short-circuit \texttt{consp}}
\begin{figure}
\begin{center}
\inputfig{fig-rewrite-6.pdf_t}
\end{center}
\end{figure}
}

\frame{\frametitle{Result after removing unreachable instructions}
\begin{figure}
\begin{center}
\inputfig{fig-rewrite-7.pdf_t}
\end{center}
\end{figure}
}

\frame{\frametitle{Final result}
\begin{figure}
\begin{center}
\inputfig{fig-rewrite-8.pdf_t}
\end{center}
\end{figure}
}

\frame{
\frametitle{Advantages to our technique}

\begin{itemize}
\item It is simple to implement.
\item Correctness is obvious, because each rewrite step preserves the
  semantics of the program.
\item In the paper, we give a proof of termination.  It works even
  when loops are replicated.
\end{itemize}

}

\frame{
\frametitle{Disadvantages}

\begin{itemize}
\item The resulting code is bigger.
\item If many, overlapping zones of liveness and redundant tests
  exist, then code size may increase exponentially.  We conjecture
  that this case is very infrequent.
\item Local rewriting is probably not the best way in terms of
  compile time performance.
\end{itemize}

}

\frame{
  \frametitle{Acknowledgments}

We would like to thank Philipp Marek for providing valuable feedback
on early versions of this paper.
}

\frame{
\frametitle{Thank you}

Questions?
}

%% \frame{\tableofcontents}
%% \bibliography{references}
%% \bibliographystyle{alpha}

\end{document}
