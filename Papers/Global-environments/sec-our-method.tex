\section{Our technique}

We suggest a \clos{}-based \emph{protocol} defining the set of
operations on a first-class environment.  This protocol contains
around $40$ generic functions.

Mainly, the protocol contains versions of \commonlisp{} environment
functions such as \texttt{fboundp}, \texttt{find-class}, etc. that
take an additional required \texttt{environment} argument.

In addition to these functions, the protocol also contains a set of
functions for accessing \emph{cells} that in most implementations
would be stored elsewhere.  Thus, a binding of a function name to a
function object contains an indirection in the form of a
\emph{function cell}.  The same holds for the binding of a variable
name (a symbol) to its \emph{global value}.  In our implementation,
these cells are ordinary \texttt{cons} cells with the \texttt{car}
containing the value of the binding, and the \texttt{cdr} contains
\texttt{nil}.

These cells are created as needed.  The first time a reference to a
function is made, the corresponding cell is created.  Compiled code
that refers to a global function will have the corresponding cell in
its run-time environment.  The cost of accessing a function at
run-time is therefore no greater in our implementation than in an
implementation that accesses the function through the symbol naming
it.
