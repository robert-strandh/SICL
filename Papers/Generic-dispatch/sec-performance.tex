\section{Performance of our method}

Unfortunately, our system (called SICL) is not yet in sufficiently
finalized to allow us to make any tests of performance.  However, we
constructed a few simulations that give us some indications of the
performance of our method compared to the method used by PCL.

In our first test, we decided to measure the time it takes for the
generic dispatch of a simple slot reader. 

First, we created a class with a single slot and a reader for that
slot like this:

\begin{verbatim}
(defclass c () ((%x :initarg :x :reader x)))  
\end{verbatim}

Next, we created a circular list of a single instance of this class:

\begin{verbatim}
(defparameter *l* (list (make-instance 'c :x 1)))

(setf (cdr *l*) *l*)
\end{verbatim}

Then, we created a function containing a loop where each iteration
reads the slot x:

\begin{verbatim}
(defun f ()
  (declare (optimize (safety 0) (speed 3) (debug 0)))
  (loop for l = *l* then (cdr l)
        repeat 1000000000
        maximize (x (car l))))
\end{verbatim}

In order to minimize overhead due to looping, we set the
\texttt{optimize} flags as shown. 

The loop calls the reader $10^9$ times.  On our computer (an x86-64
running GNU/Linux) executing this function takes around $16$ seconds.
That time includes not only calling the slot reader, but also
computing the \texttt{car} and the \texttt{cdr} of a \texttt{cons}
cell in each iteration, computing the \texttt{max}, and of course loop
iteration overhead.  Furthermore, calling the slot reader involves not
only the generic dispatch, but also checking the argument count and
some other function-call overhead.  For now, we ignore all this
overhead. 

To get some indication of the performance of our method, we need to
simulate the layout of a SICL general instance.  We do that by
defining the header as a \cl{} \texttt{struct} and by using a
\emph{simple vector} for the rack.  The definition of the header looks
like this:

\begin{verbatim}
(defstruct s class rack)
\end{verbatim}

We create a circular list of a simulated general instance as follows:

\begin{verbatim}
(defparameter *l2* 
  (let ((rack (make-array 2 :initial-contents '(10 1))))
    (list (make-s :class nil :rack rack))))

(setf (cdr *l2*) *l2*)
\end{verbatim}

Our simulated slot reader is defined like this:

\begin{verbatim}
(defun y (instance)
  (declare (optimize (safety 0) (speed 3) (debug 0)))
  (let* ((rack (s-rack instance))
         (stamp (svref rack 0)))
    (declare (type fixnum stamp))
    (if (= stamp 10)
        (svref rack 1)
        (error "1"))))

(proclaim '(notinline y))
\end{verbatim}

Finally, we define a function containing a loop that calls our
simulated slot reader:

\begin{verbatim}
(defun g ()
  (declare (optimize (safety 0) (speed 3) (debug 0)))
  (loop for l = *l2* then (cdr l)
        repeat 1000000000
        maximize (y (car l))))
\end{verbatim}

On our computer executing this function takes around $5$ seconds,
which represents a significant improvement.
