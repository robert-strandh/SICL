\section{Our technique}

To illustrate our technique, we first show a very simple version of it
in the form of the following code:

{\small\begin{verbatim}
(defun find-from-end (x list)
  (labels ((aux (x list n)
             (if (= n 1)
                 (when (eq x (car list))
                   (return-from find-from-end x))
                 (let* ((n/2 (ash n -1))
                        (half (nthcdr n/2 list)))
                   (aux x half (- n n/2))
                   (aux x list n/2)))))
    (aux x list (length list))))))
\end{verbatim}}

This function starts by computing the length of the list and then
calling the auxiliary function with the original arguments and the
length.  The auxiliary function calls \texttt{nthcdr} in order to get
a reference to about half the list it was passed.  Then it makes two
recursive calls, first with the second half of the list and then with
the firs half of the list.  The recursion terminates when the list has
a single element in it.  Then this element is compared to the argument
\texttt{x} and if they are the same,%
\footnote{We use \texttt{eq} for the comparison here, because the
  exact nature of the test is unimportant for illustrating our general
  technique.}
then the element is returned form the function.
