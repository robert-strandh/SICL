\documentclass{beamer}
\usepackage[latin1]{inputenc}
\beamertemplateshadingbackground{red!10}{blue!10}
%\usepackage{fancybox}
\usepackage{epsfig}
\usepackage{verbatim}
\usepackage{url}
%\usepackage{graphics}
%\usepackage{xcolor}
\usepackage{fancybox}
\usepackage{moreverb}
%\usepackage[all]{xy}
\usepackage{listings}
\usepackage{filecontents}
\usepackage{graphicx}

\lstset{
  language=Lisp,
  basicstyle=\scriptsize\ttfamily,
  keywordstyle={},
  commentstyle={},
  stringstyle={}}

\def\inputfig#1{\input #1}
\def\inputeps#1{\includegraphics{#1}}
\def\inputtex#1{\input #1}

\inputtex{logos.tex}

%\definecolor{ORANGE}{named}{Orange}

\definecolor{GREEN}{rgb}{0,0.8,0}
\definecolor{YELLOW}{rgb}{1,1,0}
\definecolor{ORANGE}{rgb}{1,0.647,0}
\definecolor{PURPLE}{rgb}{0.627,0.126,0.941}
\definecolor{PURPLE}{named}{purple}
\definecolor{PINK}{rgb}{1,0.412,0.706}
\definecolor{WHEAT}{rgb}{1,0.8,0.6}
\definecolor{BLUE}{rgb}{0,0,1}
\definecolor{GRAY}{named}{gray}
\definecolor{CYAN}{named}{cyan}

\newcommand{\orchid}[1]{\textcolor{Orchid}{#1}}
\newcommand{\defun}[1]{\orchid{#1}}

\newcommand{\BROWN}[1]{\textcolor{BROWN}{#1}}
\newcommand{\RED}[1]{\textcolor{red}{#1}}
\newcommand{\YELLOW}[1]{\textcolor{YELLOW}{#1}}
\newcommand{\PINK}[1]{\textcolor{PINK}{#1}}
\newcommand{\WHEAT}[1]{\textcolor{wheat}{#1}}
\newcommand{\GREEN}[1]{\textcolor{GREEN}{#1}}
\newcommand{\PURPLE}[1]{\textcolor{PURPLE}{#1}}
\newcommand{\BLACK}[1]{\textcolor{black}{#1}}
\newcommand{\WHITE}[1]{\textcolor{WHITE}{#1}}
\newcommand{\MAGENTA}[1]{\textcolor{MAGENTA}{#1}}
\newcommand{\ORANGE}[1]{\textcolor{ORANGE}{#1}}
\newcommand{\BLUE}[1]{\textcolor{BLUE}{#1}}
\newcommand{\GRAY}[1]{\textcolor{gray}{#1}}
\newcommand{\CYAN}[1]{\textcolor{cyan }{#1}}

\newcommand{\reference}[2]{\textcolor{PINK}{[#1~#2]}}
%\newcommand{\vect}[1]{\stackrel{\rightarrow}{#1}}

% Use some nice templates
\beamertemplatetransparentcovereddynamic

\newcommand{\A}{{\mathbb A}}
\newcommand{\degr}{\mathrm{deg}}

\title{\texttt{make-method-lambda} revisited}

\author{Ir�ne Durand\\Robert Strandh}
\institute{
LaBRI, University of Bordeaux
}
\date{April, 2019}

%\inputtex{macros.tex}


\begin{document}
\frame{
\resizebox{3cm}{!}{\includegraphics{Logobx.pdf}}
\hfill
\resizebox{1.5cm}{!}{\includegraphics{labri-logo.pdf}}
\titlepage
\vfill
\small{European Lisp Symposium, Genoa, Italy \hfill ELS2019}
}

\setbeamertemplate{footline}{
\vspace{-1em}
\hspace*{1ex}{~} \GRAY{\insertframenumber/\inserttotalframenumber}
}

\frame{
\frametitle{Context: The \sicl{} project}

https://github.com/robert-strandh/SICL

Several objectives:

\begin{itemize}
\item Create high-quality \emph{modules} for implementors of
  \commonlisp{} systems.
\item Improve existing techniques with respect to algorithms and data
  structures where possible.
\item Improve readability and maintainability of code.
\item Improve documentation.
\item Ultimately, create a new implementation based on these modules.
\end{itemize}
}

\frame[containsverbatim]{
\frametitle{The role of \texttt{make-method-lambda}}

Let's say we have a \texttt{defmethod} form like this:

{\footnotesize
\begin{verbatim}
(defmethod ff ((x integer))
  (1+ x))
\end{verbatim}
}

The expansion of that form is someting like this:

{\footnotesize
\begin{verbatim}
(let ((#:g001 (ensure-generic-function 'ff)))
  (add-method #:g001
    (make-instance (generic-function-method-class #:g001)
      :qualifiers '()
      :specializers (list (find-class 'integer))
      :lambda-list '(x)
      :function (lambda (arguments next-methods)
                  (flet ((next-method-p ...)
                         (call-next-method ...))
                    (appply (lambda (x) (1+ x)) arguments))))))
\end{verbatim}
}
}


\frame{
  \frametitle{Specification of \texttt{make-method-lambda}}

  SYNTAX\\
  \texttt{make-method-lambda}\\
  \quad{} {\footnotesize
    \textit{generic-function} \textit{method}
    \textit{lambda-expression} \textit{environment}
  }

  ARGUMENTS\\
  {\footnotesize
  The \textit{generic-function} argument is a generic function
  metaobject.\\
  The \textit{method} argument is a (possibly uninitialized) method
  metaobject.\\
  The \textit{lambda-expression} argument is a lambda expression.\\
  The \textit{environment} argument is the same as the
  \textbf{\&envrionment} argument to macro expasion functions.}

  VALUES\\
  {\footnotesize
  This generic function returns two values.  The first is a lambda
  expression, the second is a list of initialization arguments and
  values.}

  PURPOSE\\
  {\footnotesize
  This generic function is called to produce a lambda expression which
  can itself be used to produce a method function for a method and a
  generic function with the specified classes.  The generic function
  and method the method function will be used with are not required to
  be the given ones.  Moreover, the method metaobject may be
  uninitialized.   ...}

}

\frame[containsverbatim]{
\frametitle{Previous work}
In a paper entitled ``\texttt{make-method-lambda} considered
harmful'', Costanza and Herzeel identified a problem with
\texttt{make-method-lambda}.
\vskip 0.5cm
Consider the following two forms in the same file to be process by
\texttt{compile-file}:

{\footnotesize
\begin{verbatim}
(defgeneric foo (...)...)

(defmethod foo (...)...)
\end{verbatim}
}

}

\frame{
  \frametitle{From the \commonlisp{} standard about \texttt{defgeneric}}
  
``\texttt{defgeneric} is not required to perform any compile-time side
effects. ... An implementation may choose to store information about
the generic function for the purposes of compile-time error-checking
(such as checking the number of arguments on calls, or noting that a
definition for the function name has been seen).''
\vskip 0.5cm

In other words, the generic function is not necessarily created at
compile time, and it can be argued that it \emph{should not} be
created then.

}

\frame[containsverbatim]{
\frametitle{Previous work}
Consider again the following two forms in the same file to be process by
\texttt{compile-file}:

{\footnotesize
\begin{verbatim}
(defgeneric foo (...)...)

(defmethod foo (...)...)
\end{verbatim}
}

If the generic function is not created in the compilation environment,
then what arguments does \texttt{defmethod} pass to
\texttt{make-method-lambda}?

}

\frame{
\frametitle{Future work}


}

\frame{
  \frametitle{Acknowledgments}

}

\frame{
\frametitle{Thank you}

Questions?
}

%% \frame{\tableofcontents}
%% \bibliography{references}
%% \bibliographystyle{alpha}

\end{document}
