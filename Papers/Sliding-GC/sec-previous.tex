\section{Previous work}

\subsection{Haddon and Waite}

In 1967, Haddon and Waite \cite{Haddon:1967} designed the first
algorithm for compacting garbage collection.  The context of their
work was assumed to be an existing mark-and-sweep garbage collector
using a free list, and when that collector fails due to fragmentation,
even though there is enough total space available, their compacting
algorithm would take over and compact the heap.  The cost of their
algorithm was considered unimportant, because the alternative would be
to fail by terminating the program.

Rather than invoking a marking procedure to determine live data, they
imagined using the existing free list to determine areas of available
storage. 

An entry in their break table indicates a start address of a zones of
live data and the total amount of free space below that address.
There are many different possible variations on the exact contents of
the break table, but they are all equivalent.

They show that, if each object requires at least two words of storage,
then the total amount of free space in the entire heap is large enough
to hold the break table.  As a result, no additional space is
required for their technique to work.

\subsection{Other work}

Cohen and Nocolau \cite{Cohen:1983:CCA:69575.357226} compared several
compacting collection techniques.

Abuaiadh et al \cite{Abuaiadh:2004:EPH:1028976.1028995}.
