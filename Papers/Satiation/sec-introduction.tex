\section{Introduction}

PCL \cite{Bobrow:1986:CML:28697.28700} is a common system to include
in a \cl{} implementation.  PCL was written so that \clos{} could be
added to a pre-\clos{} Common Lisp implementation such as the one
defined in CLtL \cite{Steele:1984:CLL} without too much effort.  Even
\cl{} implementations that do not use PCL (such as ECL) include
\clos{} late in the process of building a complete system.

\sicl{}\footnote{https://github.com/robert-strandh/SICL} takes a
different approach.  With very few exceptions, \sicl{} is written in
entirely standard \cl{}, and it is designed to be bootstrapped using a
conforming \cl{} implementation, which therefore includes a complete
implementation of \clos{}.  \sicl{} takes advantage of the conforming
host by making extensive use of \clos{}.  In particular, \clos{} is
bootstrapped \emph{first}, using the host \clos{} implementation to
break circularity in definitions. 

The \sicl{} implementation of \clos{} is a truly metacircular
implementation in that very few compromises are necessary because of
bootstrapping or metastability issues.

AMOP \cite{Kiczales:1991:AMP:574212}

