\section{Previous work}

The AMOP \cite{Kiczales:1991:AMP:574212} contains a section entitled
``Living with Circularity'', which describes the essential nature of
the two kinds of issues discussed here, namely \emph{bootstrapping
  issues} and \emph{metastability issues}.  This section does not
contain a complete list of all possible issues in any implementation
of \clos{}, and probably could not contain such a list, since it would
depend on the exact organization of each particular implementation.

The section in the AMOP has two subsections, one for bootstrapping
issues and one for metastability issues.  The subsection on
bootstrapping issues is more complete.  

\subsection{Bootstrapping issues}

The section on bootstrapping issues contains two explicit issues.

The first one involves the class \texttt{standard-class} which is the
metaclass of all standard classes, including itself.  The authors
simply suggest creating this class manually.

The second issue involves the fact that generic function are required
in order to create classes, but during bootstrapping, there are no
generic functions since generic functions are instances of classes.
The technique used to handle such issues is to define ordinary
functions to contain code for essential methods, so that such
functions can be called during bootstrapping.  To avoid code
duplication, the methods defined later in the bootstrapping process
simply call those functions.  

\subsection{Metastability issues} 

The section on metastability issues also contains two issues.

The first issue involves the function \texttt{slot-value}.  As
described, the scenario does not correspond to the specification,
because the signature of the function \texttt{slot-value-using-class}
used in the scenario is different from its definition in the
specification.  Either way, the basis of the scenario is that
\texttt{slot-value} on some instance would need to access the list of
slot descriptions of the class of the instance, and that list is
contained in a slot, so that a recursive use of \texttt{slot-value}
would be required on the class of the instance.  However in a
high-performance implementation, a slot reader would not call
\texttt{slot-value}.  The reason is that \texttt{slot-value} is much
too general, so that unnecessary work would be done.  In particular, 
\texttt{slot-value} must find a slot description metaobject with a
particular name, whereas this name is already known in the slot
reader function.  Instead, in a high-performance implementation, the
slot reader would access the slot directly by location.%
\footnote{The situation is a bit more complicated due to the fact that
  the location may vary according to the exact subclass of the
  specializer of the reader method.  In fact, it can even be the case
  that the slot has different allocation in different subclasses.}

As described in the introduction, the second issue has to do with
\texttt{compute-discriminating-function}.  Again, the scenario
described is an approximation of that of a real high-performance
implementation.  Their example involves adding a method to some
generic function \texttt{F}, which would trigger the computation of a
new discriminating function for \texttt{F}.  The metastability issue
happens when \texttt{F} happens to be the function
\texttt{compute-discriminating-function}.  In that case,
\texttt{compute-discriminating-function} would be called with itself
as an argument, in which case, according to the AMOP, ``the game would
of course be over.''


%%  LocalWords:  metastability metaclass accessors specializer
%%  LocalWords:  metaobject subclasses
