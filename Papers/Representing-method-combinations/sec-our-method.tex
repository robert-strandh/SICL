\section{Our technique}

We introduce a class named \texttt{method-combination-template}.  An
instance of this class represents all method combinations with the
same \emph{name}, independent of the options.  There is a template for
\texttt{standard}, a template for \texttt{and}, etc.

A method combination, as used by the \commonlisp{} standard and by the
MOP is an instance of the class \texttt{method-combination}.  There is
not a subclass of this class for each method-combination type
mentioned in the \commonlisp{} standard.  In other words, a method
combination is a \emph{variant} of a method-combination template.

To avoid creating a new variant instance for each generic function, we
use the following technique.  We analyze the lambda list in the long
form of \texttt{define-method-combination}.  The \commonlisp{}
standard requires this lambda list to be an \emph{ordinary} lambda
list.  We assume that this lambda list contains only lexical
variables.  The result of the analysis of this lambda list is a list
of lexical variables that are available to the body of the code in the
\texttt{define-method-combination} form.  From this list, we define a
function called the \emph{variant determiner} as follows:

\begin{verbatim}
(lambda (...)
  (list v1 v2 ... vn))
\end{verbatim}

The lambda list of this function is the lambda list of the
\texttt{define-method-combination} form and \texttt{v1}, \texttt{v2},
..., \texttt{vn} are the lexical variables resulting from our analysis
of the lambda list.  Applying this function to the \emph{options}
given in the \texttt{defgeneric} form or given to the
\texttt{find-method-combination} function returns a list of objects
that uniquely determines the result of a variant, provided that it
only refers to the lexical variables in its lambda list.  We call this
list the \emph{variant signature}.

In addition to the variant-determiner function, we create a second
function as follows:

\begin{verbatim}
(lambda (<method-list> v1 v2 ... vn)
  <body>)
\end{verbatim}

where \texttt{<method-list>} is a lexical variable that contains the
list of methods to be processed by the method combination, and where
\texttt{<body>} is the expanded body of the
\texttt{define-method-combination} form.  We call this function the
\emph{effective method form function}.  Clearly, with the
restrictions that we impose, applying this function to the variant
signature has the same effect as if the original lambda list had been
used with the method-combination options given in the
\texttt{defgeneric} form or given to the
\texttt{find-method-combination} function.

We store the effective method form function in the method-combination
template.  The template contains a list of variants and each variant
contains a reference to the template as well as the variant signature.

When \texttt{find-method-combination} is called, we find the template
with the name that is passed as an argument.  Then, we apply the
variant determiner to the options, also passed as an argument in order
to obtain a variant signature.  We use the variant signature to
traverse the list of variants to determine whether any signature is
\texttt{equal} to the one just computed.  If so, the corresponding
variant is returned.  If not, a new variant is allocated, stored in
the list of variants, and then returned.

Clearly, there are some restrictions associated with our technique.  A
method-combination option can only be of certain restricted types that
allow comparisons using \texttt{equal}.  In practice, however, options
are typically symbols, so this restriction should not pose a problem.
