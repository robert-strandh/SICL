\section{Conclusions and future work}
\label{sec-conclusions}

We define a subclass \texttt{standard-method-combination} of the
specified class \texttt{method-combination}.  Method combinations
created from a method-combination type defined by the macro
\texttt{define-method-combination} are all instances of this subclass.

Our technique allows for early detection of mismatches between the
method-combination options given when a method combination is created
as a result of calling \texttt{find-method\-combination} and the lambda
list given to the invocation of \texttt{define-method-combination}.
We detect such mismatches when a new method combination is created,
but also when a method-combination type is redefined with a modified
invocation of \texttt{define-method-combination} using the name
of an existing method-combination type.

Furthermore, while a mismatch exists, our technique results in an
error being signaled whenever an attempt is made to use the faulty
method combination in order to create an effective method.

Future work includes incorporating our technique into the \sicl{} code
base.  Currently, \sicl{} does not have any data structure allowing
weak references, but such references would be desirable for the back
pointer froma method combination to the generic functions using it.
Otherwise, a memory leak would result from using \texttt{fmakunbound}
or some other operator that makes a generic function unreferenced
aside from the back pointer.
