\documentclass{sig-alternate-05-2015}
\usepackage[utf8]{inputenc}

\def\inputfig#1{\input #1}
\def\inputtex#1{\input #1}
\def\inputal#1{\input #1}
\def\inputcode#1{\input #1}

\inputtex{logos.tex}
\inputtex{refmacros.tex}
\inputtex{other-macros.tex}

\begin{document}
\setcopyright{rightsretained}
\title{Bootstrapping \commonlisp{} using \commonlisp{}}
\numberofauthors{1}
\author{\alignauthor
Robert Strandh\\
\affaddr{University of Bordeaux}\\
\affaddr{351, Cours de la Libération}\\
\affaddr{Talence, France}\\
\email{robert.strandh@u-bordeaux1.fr}}


\maketitle

\begin{abstract}
Some \commonlisp{} implementations are not bootstrapped at all, and
instead they evolve through careful modifications to an existing
\emph{image}.  Most of the remaining implementations are bootstrapped
using some lower-level language; typically \clanguage{}.  But, in most
cases, there is no profound reason for using a language other than
\commonlisp{} for creating a \commonlisp{} system, though there are
some annoying details that have to be dealt with.

We describe the bootstrapping technique used with \sicl{}, a modern
implementation of \commonlisp{}.  Though both \sicl{} and the
bootstrapping process for creating it are still being worked on, they
are sufficiently evolved that the big picture outlined in this paper
will remain valid.  Our technique uses \emph{first-class global
  environments} to isolate the host environment from the environments
required during the bootstrapping process.
\end{abstract}

\begin{CCSXML}
  <ccs2012>
  <concept>
  <concept_id>10011007.10011006.10011008.10011024.10011027</concept_id>
  <concept_desc>Software and its engineering~Control structures</concept_desc>
  <concept_significance>500</concept_significance>
  </concept>
  </ccs2012>
\end{CCSXML}

\ccsdesc[500]{Software and its engineering~Control structures}

\printccsdesc

\keywords{\clos{}, \commonlisp{}, Iteration}

\inputtex{sec-introduction.tex}
\inputtex{sec-previous.tex}
\inputtex{sec-our-method.tex}
\inputtex{sec-benefits.tex}
\inputtex{sec-conclusions.tex}
\inputtex{sec-acknowledgments.tex}

\bibliographystyle{abbrv}
\bibliography{bootstrapping}
\end{document}
