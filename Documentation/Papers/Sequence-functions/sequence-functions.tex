\documentclass{acm_proc_article-sp}
\usepackage[utf8]{inputenc}
\usepackage{color}

\def\inputfig#1{\input #1}
\def\inputtex#1{\input #1}
\def\inputal#1{\input #1}
\def\inputcode#1{\input #1}

\inputtex{logos.tex}
\inputtex{refmacros.tex}
\inputtex{other-macros.tex}

\begin{document}
\title{Fast, Maintainable, and Portable Sequence Functions}
\numberofauthors{1}
\author{\alignauthor
Irène Durand\\
Robert Strandh\\
\affaddr{University of Bordeaux}\\
\affaddr{351, Cours de la Libération}\\
\affaddr{Talence, France}\\
\email{irene.durand@u-bordeaux.fr}
\email{robert.strandh@u-bordeaux.fr}}

\toappear{Permission to make digital or hard copies of all or part of
  this work for personal or classroom use is granted without fee
  provided that copies are not made or distributed for profit or
  commercial advantage and that copies bear this notice and the full
  citation on the first page. Copyrights for components of this work
  owned by others than the author(s) must be honored. Abstracting with
  credit is permitted. To copy otherwise, or republish, to post on
  servers or to redistribute to lists, requires prior specific
  permission and/or a fee. Request permissions from
  Permissions@acm.org.

  ELS '17, April 3 -- 6 2017, Brussels, Belgium
  Copyright is held by the owner/author(s). %Publication rights licensed to ACM.
%  ACM 978-1-4503-2931-6/14/08\$15.00.
%  http://dx.doi.org/10.1145/2635648.2635654
}

\maketitle

\begin{abstract}
The \commonlisp{} sequence functions are challenging to implement
because of the numerous cases that need to be taken into account
according to the keyword arguments given and the type of the sequence
argument, including the element type when the sequence is a vector.

For the resulting code to be fast, the different cases need to be
handled separately, but doing so may make the code hard to understand
and maintain.  Writing tests that cover all cases may also be
difficult.

In this paper, we present a technique that relies on a good compiler
to optimize each separate case according to the information available
to it with respect to types and values of keyword arguments.  Our
technique uses a few custom macros that duplicate a general
implementation of the body of a sequence function.  The compiler then
specializes that body in different ways for each copy.
\end{abstract}

\category{D.3.4}{Programming Languages}{Processors}
[Code generation, Optimization, Run-time environments]

\terms{Algorithms, Languages, Performance}

\keywords{\clos{}, \commonlisp{}, Generic dispatch, Method dispatch}

\inputtex{sec-introduction.tex}
\inputtex{sec-previous.tex}
\inputtex{sec-our-method.tex}
\inputtex{sec-performance.tex}
\inputtex{sec-conclusions.tex}
\inputtex{sec-acknowledgements.tex}

\bibliographystyle{abbrv}
\bibliography{sequence-functions}
\end{document}
