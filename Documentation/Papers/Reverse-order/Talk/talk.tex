\documentclass{beamer}
\usepackage[latin1]{inputenc}
\beamertemplateshadingbackground{red!10}{blue!10}
%\usepackage{fancybox}
\usepackage{epsfig}
\usepackage{verbatim}
\usepackage{url}
%\usepackage{graphics}
%\usepackage{xcolor}
\usepackage{fancybox}
\usepackage{moreverb}
%\usepackage[all]{xy}
\usepackage{listings}
\usepackage{filecontents}
\usepackage{graphicx}

\lstset{
  language=Lisp,
  basicstyle=\scriptsize\ttfamily,
  keywordstyle={},
  commentstyle={},
  stringstyle={}}

\def\inputfig#1{\input #1}
\def\inputeps#1{\includegraphics{#1}}
\def\inputtex#1{\input #1}

\inputtex{logos.tex}

%\definecolor{ORANGE}{named}{Orange}

\definecolor{GREEN}{rgb}{0,0.8,0}
\definecolor{YELLOW}{rgb}{1,1,0}
\definecolor{ORANGE}{rgb}{1,0.647,0}
\definecolor{PURPLE}{rgb}{0.627,0.126,0.941}
\definecolor{PURPLE}{named}{purple}
\definecolor{PINK}{rgb}{1,0.412,0.706}
\definecolor{WHEAT}{rgb}{1,0.8,0.6}
\definecolor{BLUE}{rgb}{0,0,1}
\definecolor{GRAY}{named}{gray}
\definecolor{CYAN}{named}{cyan}

\newcommand{\orchid}[1]{\textcolor{Orchid}{#1}}
\newcommand{\defun}[1]{\orchid{#1}}

\newcommand{\BROWN}[1]{\textcolor{BROWN}{#1}}
\newcommand{\RED}[1]{\textcolor{red}{#1}}
\newcommand{\YELLOW}[1]{\textcolor{YELLOW}{#1}}
\newcommand{\PINK}[1]{\textcolor{PINK}{#1}}
\newcommand{\WHEAT}[1]{\textcolor{wheat}{#1}}
\newcommand{\GREEN}[1]{\textcolor{GREEN}{#1}}
\newcommand{\PURPLE}[1]{\textcolor{PURPLE}{#1}}
\newcommand{\BLACK}[1]{\textcolor{black}{#1}}
\newcommand{\WHITE}[1]{\textcolor{WHITE}{#1}}
\newcommand{\MAGENTA}[1]{\textcolor{MAGENTA}{#1}}
\newcommand{\ORANGE}[1]{\textcolor{ORANGE}{#1}}
\newcommand{\BLUE}[1]{\textcolor{BLUE}{#1}}
\newcommand{\GRAY}[1]{\textcolor{gray}{#1}}
\newcommand{\CYAN}[1]{\textcolor{cyan }{#1}}

\newcommand{\reference}[2]{\textcolor{PINK}{[#1~#2]}}
%\newcommand{\vect}[1]{\stackrel{\rightarrow}{#1}}

% Use some nice templates
\beamertemplatetransparentcovereddynamic

\newcommand{\A}{{\mathbb A}}
\newcommand{\degr}{\mathrm{deg}}

\title{Processing List Elements in Reverse Order}

\author{Ir�ne Durand and Robert Strandh}
\institute{
LaBRI, University of Bordeaux
}
\date{April, 2015}

%\inputtex{macros.tex}


\begin{document}
\frame{
\resizebox{3cm}{!}{\includegraphics{Logobx.pdf}}
\hfill
\resizebox{1.5cm}{!}{\includegraphics{labri-logo.pdf}}
\titlepage
\vfill
\small{European Lisp Symposium, London, UK \hfill ELS2015}
}

\setbeamertemplate{footline}{
\vspace{-1em}
\hspace*{1ex}{~} \GRAY{\insertframenumber/\inserttotalframenumber}
}

\frame{
\frametitle{Context: The \sicl{} project}

https://github.com/robert-strandh/SICL

Several objectives:

\begin{itemize}
\item Create high-quality \emph{modules} for implementors of
  \commonlisp{} systems.
\item Improve existing techniques with respect to algorithms and data
  structures where possible.
\item Improve readability and maintainability of code.
\item Improve documentation.
\item Ultimately, create a new implementation based on these modules.
\end{itemize}
}

\frame{
\frametitle{Processing list elements in reverse order}

\begin{itemize}
\item Several \commonlisp{} operators accept the \texttt{:from-end}
  keyword argument.
\item In most cases, an implementation is still allowed to process
  from beginning to end, e.g., \texttt{find} and \texttt{position}.
\item Two operators are \emph{required} to process from the end, namely
  \texttt{count} and \texttt{reduce}.
\end{itemize}

In this talk, we will exclusively refer to the \texttt{count}
operator, but the same argument is true for \texttt{reduce}.
}

\frame[containsverbatim]{
\frametitle{The \commonlisp{} \texttt{count} operator}

\begin{verbatim}
count item sequence
      &key from-end start end key test test-not => n
\end{verbatim}


\begin{quotation}
  The from-end has no direct effect on the result.
  However, if from-end is true, the elements of sequence
  will be supplied as arguments to the test, test-not,
  and key in reverse order, which may change the side-effects,
  if any, of those functions.
\end{quotation}
  
}

\frame{
\frametitle{Existing implementations of \texttt{count}}

\sbcl{}, \ccl{}, and \lispworks{} all implement the behavior of
\texttt{count} by first calling \texttt{reverse} on the list.

Inconveniences:

\begin{itemize}
\item For long lists, an unreasonable amount of heap space is used,
  and it is possible to run out of space.
\item Allocating list elements \emph{may} be costly.
\item The garbage collector will run more often.
\end{itemize}

}

\frame{
\frametitle{Basis of our technique}

\begin{itemize}
\item Use available stack space.
\item Traverse the list multiple times if necessary.
\end{itemize}

Most of the talk will discuss near-portable techniques.
\vskip 0.25cm
Toward the end, more efficient non-portable techniques.
\vskip 0.25cm
The main performance factor of our technique is the number of
\texttt{cdr} operations.  For that reason, we will consider that
\texttt{:test} is \texttt{eq} and that \texttt{:key} is
\texttt{identity}.  Other cases will be more favorable to our
technique.
}

\frame[containsverbatim]{
\frametitle{Basic recursive technique}

\begin{verbatim}
(defun recursive-count (x list)
  (if (endp list)
      0
      (+ (recursive-count x (cdr list))
         (if (eq (car list) x) 1 0))))
\end{verbatim}

This technique uses \emph{stack} space rather than heap space.\\
Problem: stack space is limited, so it may fail.
\vskip 0.25cm
Otherwise, it is very fast and does not solicit the garbage
collector.

}

\frame[containsverbatim]{
\frametitle{Avoiding stack overflow}

\begin{verbatim}
(defun count-from-end (x list)
  (labels ((aux (x list n)
             (cond ((= n 0) 0)
                   ((= n 1)
                    (if (eq x (car list)) 1 0))
                   (t (let* ((m (ash n -1))
                             (suffix (nthcdr m list)))
                        (+ (aux x suffix (- n m))
                           (aux x list m)))))))
    (aux x list (length list))))))
\end{verbatim}

Number of \texttt{cdr} operations $F(n)$ on a list of length $n$ is
approximately $F(n) \approx n\thinspace (1 +
\frac{1}{2}\mathsf{lb}\thinspace n)$.
\vskip 0.25cm
Unacceptable when $n$ is very large.
\vskip 0.25cm
Stack depth is only $\mathsf{lb}\thinspace n$.
}

\frame[containsverbatim]{
\frametitle{Using more stack space}

\begin{verbatim}
(defun count-from-end-2 (x list)
  (labels ((recursive (x list n)
             (if (zerop n)
                 0
                 (+ (recursive x (cdr list) (1- n))
                    (if (eq x (car list)) 1 0))))
           (aux (x list n)
             (if (<= n 10000)
                 (recursive x list n)
                 (let* ((m (ash n -1))
                        (suffix (nthcdr m list)))
                   (+ (aux x suffix (- n m))
                      (aux x list m))))))
    (aux x list (length list))))
\end{verbatim}
}

\frame[containsverbatim]{
\frametitle{Using more stack space}

\begin{verbatim}
(defun count-from-end-2 (x list)
  (labels ((recursive (x list n)
             (if (zerop n)
                 0
                 (+ (recursive x (cdr list) (1- n))
                    (if (eq x (car list)) 1 0))))
           (aux (x list n)
             (if (<= n 10000)
                 (recursive x list n)
                 (let* ((m 10000)
                        (suffix (nthcdr m list)))
                   (+ (aux x suffix (- n m))
                      (aux x list m))))))
    (aux x list (length list))))
\end{verbatim}
}

\frame[containsverbatim]{
\frametitle{Using more stack space}

\begin{verbatim}
(defun count-from-end-2 (x list)
  (labels ((recursive (x list n)
             (if (zerop n)
                 0
                 (+ (recursive x (cdr list) (1- n))
                    (if (eq x (car list)) 1 0))))
           (aux (x list n)
             (if (<= n 10000)
                 (recursive x list n)
                 (let* ((m 10000)
                        (suffix (nthcdr m list)))
                   (+ (aux x suffix (- n m))
                      (recursive x list m))))))
    (aux x list (length list))))
\end{verbatim}
}

\frame[containsverbatim]{
\frametitle{Using more stack space}

{\small
\begin{verbatim}
(defun count-from-end-2 (x list)
  (labels ((recursive (x list n)
             ...)
           (aux1 (x list n)
             (if (<= n 10000)
                 (recursive x list n)
                 (let* ((m 10000)
                        (suffix (nthcdr m list)))
                   (+ (aux1 x suffix (- n m))
                      (recursive x list m)))))
           (aux2 (x list n)
             (if (<= n 100000000)
                 (aux1 x list n)
                 (let* ((m (ash n -1))
                        (suffix (nthcdr m list)))
                   (+ (aux2 x suffix (- n m))
                      (aux2 x list m))))))
    (aux2 x list (length list))))
\end{verbatim}
}
}

\frame{
  \frametitle{Benchmark of three versions}

  \begin{itemize}
  \item v0 Native \texttt{count} with \texttt{:from-end t}.
  \item v1 Using \texttt{reverse} then \texttt{count}.
  \item v7 Our best version.
  \end{itemize}
}

\frame{
  \frametitle{Comparison of the behavior of the three versions}
\begin{tabular}{|c|c|c|c|c|c|c|}
\hline
\multicolumn{3}{|c|}{System characteristics}  & \multicolumn{3}{|c|}{Time in ms}\\ \hline
Impl & Version        & Processor  & v0   & v1   & v7   \\ \hline
LispWorks  & 6.1.1    & Intel Core & 0.20 & 0.18 & 0.14 \\ \hline
Clozure CL & 1.10     & Intel Xeon & 1.93 & 1.79 & 0.15 \\ \hline
Clozure CL & 1.10-dev & AMD FX     & 1.77 & 1.63 & 0.15 \\ \hline
SBCL       & 1.2.8    & Intel Xeon & 0.51 & 0.27 & 0.22 \\ \hline
ABCL       & 1.3.1    & Intel Xeon & 1.13 & 0.22 & 0.34 \\ \hline
CLISP      & 2.49     & X86\_64    & 1.15 & 1.14 & 0.87 \\ \hline
ECL        & 13.5.1   & ?          & 0.69 & 0.41 & 0.36 \\ \hline
SBCL       & 1.2.7    & Intel Core & 0.36 & 0.38 & 0.25 \\ \hline
\end{tabular}
}

\frame{
  \frametitle{Comparison of the behavior of the three versions}
  \begin{center}
    \includegraphics{v0-v1-v7.eps}
  \end{center}
}

\frame{
  \frametitle{Doing better with implementation-specific techniques}

So far, our technique is mostly portable (aside from computing
available stack space).
\vskip 0.25cm
Observations:

\begin{itemize}
\item Each recursive call takes many words in each stack frame for
  frame pointer, return address, saved registers, etc.
\item We use \texttt{nthcdr} which must test for \texttt{cons} cells,
  but no such test is needed.
\end{itemize}

}

\frame[containsverbatim]{
\frametitle{Replacing recursive technique}

\begin{verbatim}
(defun low-level-reverse-count (item list length)
  (loop for rest = list then (cdr rest)
        repeat length
        do (push-on-stack (car rest)))
  (loop repeat length
        count (eq item (pop-from-stack))))
\end{verbatim}
}

\frame{
  \frametitle{Improvement with implementation-specific technique}

\begin{itemize}
\item In a typical implementation, we can process more than one
  hundred thousand elements without any function calls.
\item No need to test whether element is a \texttt{cons} cell.
\item We can process lists of up to $10$ billion elements with only
  $3$ \texttt{cdr} operations per element, $2$ of which need not be
  checked.
\end{itemize}

Result: On \sbcl{}, speed almost the same as counting from the
beginning of the list.

}

\frame{
\frametitle{Future work}

\begin{itemize}
\item Apply our technique to operators where the implementation has a
  choice of direction of traversal, in particular \texttt{find} and
  \texttt{position}.
\item Devise benchmarks for these operators.  Additional parameters
  are necessary, such as expected position of last element that
  \emph{passes the test}.
\item Apply the idea of using available stack space to other domains.
\end{itemize}
}

\frame{
  \frametitle{Acknowledgments}

We would like to thank Pascal Bourguignon for reading and commenting
on an early draft of this paper.  We would also like to thank Alastair
Bridgewater for helping us with the foreign-function interface of
\sbcl{}.  Also, we would like to thank Pascal Bourguignon, Alastair
Bridgewater, James Kalenius, Steven Styer, Nicolas Hafner, and Eric
Lind for helping us run benchmarks on platforms that are unavailable
to us.  Finally, we would like to thank Martin Simmons at \lispworks{}
technical support for giving us the information we needed in order to
explain the performance of our technique on the \lispworks{}
\commonlisp{} implementation.
}

\frame{
\frametitle{Thank you!}
Questions?
}

%% \frame{\tableofcontents}
%% \bibliography{references}
%% \bibliographystyle{alpha}

\end{document}
