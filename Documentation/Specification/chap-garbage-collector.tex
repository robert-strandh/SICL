\chapter{Garbage collector}

To fully appreciate the contents of this chapter, the reader should
have some basic knowledge of the usual techniques for garbage
collection.  We recommend ``The Garbage Collection Handbook''
\cite{Jones:2011:GCH:2025255} to acquire such basic knowledge.  We
also recommend Paul Wilson's excellent survey paper
\cite{Wilson:1992:UGC:645648.664824}.

We think it would be good to use a per-thread nursery combined with a
global allocator for older objects.  This technique has been published
by Doligez and Leroy (see \cite{Doligez:1993:CGG:158511.158611}) and
perhaps in other papers as well.

In \refChap{chap-data-representation} we suggest a data representation
where every heap allocated object has a two-word header object.  If it
is a \texttt{cons} cell, then that is the entire object.  For other
heap-allocated objects (called \emph{genral instances}, the first of
the two words is a tagged pointer to the class object, and the second
of the two words is a raw pointer to the \emph{rack}.  In
this chapter, we describe the consequences of this suggested
representation to the garbage collector.

\section{Global collector}
No work has been done yet on the global collector.  

The global collector is a concurrent collector, i.e., it runs in
parallel with the mutator threads.  With modern processors, it is
probably practical to assign at least one core more or less
permanently to the global collector.  According to current thinking,
the global collector will be a combination of a mark-and-sweep
collector and a mark-and-compact collector as follows.  

We define a global heap where \texttt{cons} cells and the
\emph{headers} of general instances are allocated from the
\emph{beginning} of the heap and \emph{racks} are allocated from the
\emph{end}.  Since all general instances have a two-word header and
\texttt{cons} cells consist of two words as well, we can use a
mark-and-sweep collector for these objects without suffering any
fragmentation.  The advantage of a mark-and-sweep collector is that
objects will never move, which is an advantage when they are used as
keys in hash tables and when they are used to communicate with
code in foreign languages that assume that an address of an object is
fixed once and for all.

For the racks, we plan to use the sliding collector created by Kermany
et all \cite{Kermany:2006:CCI:1133981.1134023}.  Their collector uses
very little additional memory because, although it define two semi
spaces, nearly half of the pages will always be \emph{virtual}, i.e.,
without any backing store.  Furthermore, their collector is
\emph{parallel}, meaning that several threads can collect
simultaneously and \emph{concurrent}, i.e., it can work in parallel
with mutator threads.  We do not plan to use this collector for the
nurseries, because it uses memory-mapping techniques which incur
additional cost.

Even though racks more around, objects do not, simply because the
objects \emph{is} the header, and headers do not move.  

\section{Nursery collector}
The directory \texttt{Code/Garbage-collector} contains some code that
uses a different data representation from what we now suggest.  

For the nursery, we suggest using a copying collector to manage small
(a few megabytes) linear space.  Instead of promoting objects that
survive a collection, we suggest using a \emph{sliding collector} in
the nursery.  Such a collector gives a very precise idea of the age of
different objects, so objects would always be promoted in the order of
the oldest to the youngest.  This technique avoids the problem where
the allocation of some intermediate objects is immediately followed by
a collection, so these objects are promoted even though they are
likely to die soon after the collection.  In a sliding collector,
promotion will happen only when a collection leaves insufficient space
in the nursery, at which point only the number of objects required to
free up enough memory would be promoted, and in the strict order of
oldest to youngest.

We suggest allocating header objects from the beginning of the nursery
and racks from the end.  A garbage collection is triggered
when the two free pointers meet.  We preserve the relative order both
of header objects and of racks. 

In the \emph{mark phase}, a single mark bit per header object is
used.  Header objects and racks are scanned, but only
header objects are marked.  

In the \emph{header compaction phase}, live header objects are compacted
toward the beginning of the heap.  A \emph{source} and a \emph{target}
pointer follow blocks in the bitmap and live words are moved. 

In the \emph{table build phase}, the bitmap is used to construct an
\emph{offset table} at the end of the lower part of the heap.  The
offset table contains a sequence of pairs $<a,o>$ meaning that a
pointer with an address less than or equal to $a$ (and greater than
the corresponding field of the previous pair) should be adjusted by
offset $o$.  There is always enough room for this table, because it
can contain at most as many entries as there are dead header objects.  

In the \emph{adjust phase}, header objects and racks are
scanned, and fields containing pointers are adjusted according to the
offset table.  The offset table is searched using binary search,
except that a simple caching scheme is used to avoid a full binary
search in nearly all cases. 

In the \emph{rack compaction phase}, header objects are scanned in
order of increasing addresses, and the corresponding racks
are compacted towards the end of the heap.

\section{Promotion}

Objects are typically promoted when the space recovered in the nursery
as a result of a thread-local garbage collection is deemed too small.%
\footnote{Here ``too small'' means that another garbage collection
  would be triggered very soon again, which is not desirable.}  When
this happens, a \emph{prefix} of the nursery (and the associated
racks in the suffix) is moved to the space managed by the
global collector.  Then, pointers are adjusted according to the size
of the prefix and the suffix. 

Promotion could happen for reasons other than age.  Objects that are
too large for the nursery would be allocated directly in the global
allocator.  Objects to which references from foreign code are
about to be created would first be promoted to the global collector
where they would no longer move.  The same is true for an object that
is about to be used as key in an \texttt{eq} hash table.  Objects that
can be expected to have a significant life span (such as symbols)
might also be allocated directly in the global collector. 

\section{Implementation}

In most systems, the garbage collector is implemented in some language
other than \commonlisp{}.  However, we imagine using \commonlisp{} together with the
primitives described in \refChap{chap-low-level-primitives} to
implement it.

%%  LocalWords:  mutator
