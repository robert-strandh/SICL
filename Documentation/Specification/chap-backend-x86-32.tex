\chapter{x86-32}
\label{chap-backend-x86}

A \emph{function} contains:

\begin{itemize}
\item A static environment.
\item The entry point of the function as a raw machine address.  Since
  entry points are word aligned, this value looks like a fixnum.
\end{itemize}

\section{Calling conventions}

\refFig{fig-x86-32-stack-frame} shows the layout of a stack frame.

\begin{figure}
\begin{center}
\inputfig{fig-x86-32-stack-frame.pdf_t}
\end{center}
\caption{\label{fig-x86-32-stack-frame}
Stack frame for the x86-32 backend.}
\end{figure}

Normal call to an external function:

\begin{enumerate}
\item (caller) Subtract $4 \cdot (N + 4)$ from \texttt{ESP} where $N$
  is the number of arguments to be passed.  The number $4$ to be added
  to $N$ is the number of words required for the caller \texttt{EBP},
  the return address, the static environment, and the dynamic environment.
\item (caller) Store the arguments to be passed in
  \texttt{[ESP$+ 4 \cdot 0$]},
  \texttt{[ESP$+ 4 \cdot 1$]},
  $\ldots$
  \texttt{[ESP$+ 4 \cdot (N - 1)$]}.
  Copy the dynamic environment in
  \texttt{[EBP$-4 \cdot 3$]} to
  \texttt{[ESP$+4 \cdot N$]}.
  Store the static environment of the function to be called in
  \texttt{[ESP$+4 \cdot (N + 1)$]}.
  Store \texttt{EBP} in
  \texttt{[ESP$+4 \cdot (N + 3)$]}.
  These operations can be done in any order.
\item (caller) Load the entry point of the function into a scratch
  register.
\item (caller) Store \texttt{ESP$+4 \cdot (N + 3)$} in \texttt{EBP},
  establishing the stack frame of the callee.
\item (caller) Use the \texttt{CALL} instruction to call the address
  previously computed.  The return address is then pushed on the
  stack.
  \begin{enumerate}
  \item (callee) Pop the return address off of the top of the stack,
    and store it in \texttt{[EBP$- 4 \cdot 1$]}.  The code for this
    step has a fixed size and can be skipped by a caller doing a tail
    call.
  \item (callee) Do some useful computation.
  \item (callee) Store \texttt{EBP$- 4 \cdot 1$} in \texttt{ESP},
    deleting the current stack frame.
  \item (callee) Use the \texttt{RET} instruction to return to the
    caller.
  \end{enumerate}
\item (caller) Pop \texttt{EBP}, restoring the stack frame as it was
  before the call.
\end{enumerate}

By using a \texttt{CALL}/\texttt{RET} pair instead of (say) the caller
storing the return address in its final place using some other method,
we make sure that the predictor for the return address of the
processor makes the right guess about the eventual address to be
used.

Tail call to an external function:

\begin{enumerate}
\item Make sure \texttt{EBP}$-$\texttt{ESP} is at least $4 \cdot (N +
  4)$ where $N$ is the number of arguments to be passed.  This
  difference is typically greater, because temporary space might be
  needed in order to compute the arguments.
\item Store the arguments in
  \texttt{[EBP$- 4 \cdot 4$]},
  \texttt{[EBP$- 4 \cdot 5$]},
  $\ldots$
  \texttt{[EBP$- 4 \cdot (N + 3$]},
  from left to right.
  Restore the original dynamic environment in
  \texttt{[EBP$- 4 \cdot 3$]}
  if it was modified.  Store the static environment of the callee in
  \texttt{[EBP$- 4 \cdot 2$]}.
\item Load the entry point into a scratch register as above.
\item Use the JMP instruction to jump the address previously computed.
  Nothing is pushed on the stack.
  \begin{enumerate}
  \item (callee) Do some useful computation.
  \item (callee) Store \texttt{EBP$- 4 \cdot 1$} in \texttt{ESP},
    deleting the current stack frame.
  \item (callee) Use the \texttt{RET} instruction to return to the
    caller's caller.
  \end{enumerate}
\end{enumerate}

The \hs{} says:%
\footnote{See section 3.2.2.3 in the \hs{}.}
``A call within a file to a named function that is defined in the same
file refers to that function, unless that function has been declared
\texttt{notinline}''.  We take advantage of this fact by optimizing
such calls.  For instance, in most cases it is not necessary to check
the argument count.  Furthermore, the compiler generates a single code
object for all function in one file.  The entry point of the callee
can therefore be computed by the compiler.

Normal call to a top-level internal function (i.e., a function with
the same code object as the caller):

\begin{enumerate}
\item (caller) Subtract $4 \cdot (N + 4)$ from \texttt{ESP} where $N$
  is the number of arguments to be passed.
\item (caller) Store the arguments to be passed in
  \texttt{[ESP$+ 4 \cdot 0$]},
  \texttt{[ESP$+ 4 \cdot 1$]},
  $\ldots$
  \texttt{[ESP$+ 4 \cdot (N - 1)$]}.
  Copy the dynamic environment in
  \texttt{[EBP$-4 \cdot 3$]} to
  \texttt{[ESP$+4 \cdot N$]}.
  Compute the static environment of the callee as a suffix of the
  current static environment and 
  store it in
  \texttt{[ESP$+4 \cdot (N + 1)$]}.
  Store \texttt{EBP} in
  \texttt{[ESP$+4 \cdot (N + 3)$]}.
  These operations can be done in any order.
\item (caller) Store \texttt{ESP$+4 \cdot (N + 3)$} in \texttt{EBP},
  establishing the stack frame of the callee.
\item (caller) Use the \texttt{CALL} instruction to make a relative
  call to a code snippet still within the caller.  That code snippet pops
  the return address off of the top of the stack,
  and stores it in \texttt{[EBP$- 4 \cdot 1$]}.  Finally, it uses the
  \texttt{JMP} instruction to jump to an appropriate location in the
  callee.  That appropriate location is typically after some initial
  code that checks for argument count, etc.
  \begin{enumerate}
  \item (callee) Do some useful computation.
  \item (callee) Store \texttt{EBP$- 4 \cdot 1$} in \texttt{ESP},
    deleting the current stack frame.
  \item (callee) Use the \texttt{RET} instruction to return to the
    caller.
  \end{enumerate}
\item (caller) Pop \texttt{EBP}, restoring the stack frame as it was
  before the call.
\end{enumerate}

Tail call to a top-level internal function internal function (i.e., a
function with the same code object as the caller):

\begin{enumerate}
\item Make sure \texttt{EBP}$-$\texttt{ESP} is at least $4 \cdot (N +
  4)$ where $N$ is the number of arguments to be passed.  This
  difference is typically greater, because temporary space might be
  needed in order to compute the arguments.
\item Store the arguments in
  \texttt{[EBP$- 4 \cdot 4$]},
  \texttt{[EBP$- 4 \cdot 5$]},
  $\ldots$
  \texttt{[EBP$- 4 \cdot (N + 3$]},
  from left to right.
  Restore the original dynamic environment in
  \texttt{[EBP$- 4 \cdot 3$]}
  if it was modified.  Compute the static environment of the callee as
  a suffix of the current static environment and store it in
  \texttt{[EBP$- 4 \cdot 2$]}.
\item Use the JMP instruction to jump to an appropriate location in
  the callee.  That appropriate location is typically after some
  initial code that checks for argument count, etc.  Nothing is pushed
  on the stack.
  \begin{enumerate}
  \item (callee) Do some useful computation.
  \item (callee) Store \texttt{EBP$- 4 \cdot 1$} in \texttt{ESP},
    deleting the current stack frame.
  \item (callee) Use the \texttt{RET} instruction to return to the
    caller's caller.
  \end{enumerate}
\end{enumerate}

The calling conventions above are optimized for functions with no
\&rest and no \&key parameters, and with parameters having dynamic
extent.  That case is very common, and the use of compiler macros
makes it even more common that what is apparent from reading source
code, because many calls to functions with \&rest and \&key parameters
can be replaced by calls to specialized functions with only required
parameters.  For this special case, the arguments supplied by a caller
are already in their ``home'' position on the stack.  When \&rest and
\&key parameters are present, a possibly fairly complex process must
traverse the arguments and establish a local environment.  Parameters
with dynamic extent can be allocated on the stack, whereas parameters
with indefinite extent must be placed in heap-allocated objects.
\seesec{backends-x86-32-static-environment}  The presence of
\&optional parameters represents an intermediate case.  In most cases
the default values of \&optional parameters are simple constants.  In
that case, the callee can simply push additional values on the stack
(provided the corresponding parameter has dynamic extent).  But
default values may require the evaluation of arbitrary expressions,
and those expressions might refer to the values of required parameters
or of \&optional arguments further left in the lambda list.  Such
expression can also refer to local variables in enclosing
environments.

The first return value is passed in \texttt{EAX}, and the number of
return values is passed in \texttt{EBX} as a \texttt{fixnum}.  If the
number of return values is $0$, then the callee must make sure
\texttt{EAX} contains \texttt{NIL} when it returns.  This convention
makes it unnecessary for the caller that expects a single return value
to check the count.  Furthermore, this situation is probably the most
common one, so that in almost all cases, checking the number of return
values becomes unnecessary.

\section{Use of the static environment}
\label{backends-x86-32-static-environment}

Variables that are captured by a closure with indefinite extent can
not be allocated in the stack frame.  Such variables are allocated in
the static environment.  The static environment is a list of levels,
where each level is a simple vector.  This list is stored in the stack
at \texttt{[EBP$-4 \cdot 2$]}.  A function that accesses local
variables of a containing function has the captured environment stored
as part of the object representing the function.  To use captured
variables, the code for the function copies the environment to the
stack at \texttt{[EBP$-4 \cdot 2$]}.  If the environment needs to be
extended, then it is done by pushing another level onto the list.

\section{Use of the dynamic environment}

The dynamic environment is a simply linked sequence of entries
allocated in the stack rather than on the heap.  The following entry
types exist:

\begin{itemize}
  \item An entry representing the binding of a special variable.  This
    entry contains two fields; the symbol to be bound and the value.

  \item An entry representing a \texttt{CATCH} tag.  The entry
    contains two fields; the \texttt{CATCH} tag and the value of
    \texttt{EBP} in the stack frame to return from.

  \item An entry representing a \texttt{BLOCK} or a \texttt{TAGBODY}.
    It is similar to the entry representing a \texttt{CATCH} tag.  The
    entry contains three fields; two fields representing a 64-bit time
    stamp and one field containing the value of ebp in the stack frame
    to return from.

  \item An entry representing an \texttt{UNWIND-PROTECT} form.  This
    entry contains a thunk containing the cleanup forms of the
    \texttt{UNWIND-PROTECT} form.
\end{itemize}

In addition to their own fields, each entry contains a pointer to the
next entry in the sequence, and a field indicating what type of entry
it is.

Of the three types of entries above, the \texttt{CATCH} and the
\texttt{BLOCK}/\texttt{TAGBODY} entries represent exit points.  An
exit point can be marked as "expired" or "invalid" by storing 0 as the
ebp value in the entry.

\texttt{CATCH} is implemented as a call to a function.  This function
establishes a \texttt{CATCH} tag and calls a thunk containing the body
of the \texttt{CATCH} form.  The \texttt{CATCH} tag is established by
allocating (as dynamic local data on the stack) an entry of the second
type and making \texttt{[EBP$-4 \cdot 4$]} point to it as shown in
\refFig{fig-x86-32-catch}

\begin{figure}
\begin{center}
\inputfig{fig-x86-32-catch.pdf_t}
\end{center}
\caption{\label{fig-x86-32-catch}
\texttt{CATCH} tag for the x86-32 backend.}
\end{figure}

THROW searches the dynamic environment for an entry with the right
\texttt{CATCH} tag.  If one is found and it is valid (as indicated by
the stored ebp different from 0), then the target to which control is
tranfered is the return address in the stack frame indicated by the
stored ebp.

A \texttt{BLOCK} form may establish an exit point.  In the most
general case, a RETURN-FROM is executed from a function
lexically-enclosed inside the \texttt{BLOCK} with an arbitrary number
of intervening stack frames.  When this is the case, \texttt{BLOCK} is
implemented as a call to a function, in a way similar to the way
\texttt{CATCH} is implemented.  The function establishes a
\texttt{BLOCK}/\texttt{TAGBODY} entry with a fresh time stamp.  The
time stamp is also stored in a lexical varible in the static
environment.  \refFig{fig-x86-32-block-tag} shows this situation.

\begin{figure}
\begin{center}
\inputfig{fig-x86-32-block-tag.pdf_t}
\end{center}
\caption{\label{fig-x86-32-block-tag}
\texttt{BLOCK} tag for the x86-32 backend.}
\end{figure}

RETURN-FROM consults the time stamp in the lexical variable and
searches the dynamic environment for a corresponding time stamp in a
\texttt{BLOCK}/\texttt{TAGBODY} entry.  If one is found and it is
valid (as indicated by the stored ebp different from 0), then the
target to which control is tranfered is the return address in the
stack frame indicated by the stored ebp.

A \texttt{TAGBODY} may establish several exit points.  In the most
general case, a \texttt{GO} is executed from a function
lexically-enclosed inside the \texttt{TAGBODY} with an arbitrary
number of intervening stack frames.  When this is the case,
\texttt{TAGBODY} is implemented as a call to a function, in a way
similar to the way \texttt{BLOCK} is implemented.  The function
establishes a \texttt{BLOCK}/\texttt{TAGBODY} entry with a fresh time
stamp.  The time stamp is also stored in a lexical varible in the
static environment.  For each tag that is the target of a \texttt{GO},
an address to wich control is transfered is also stored in a lexical
environment.

\texttt{GO} consults the time stamp in the lexical variable and
searches the dynamic environment for a corresponding time stamp in a
\texttt{BLOCK}/\texttt{TAGBODY} entry.  If one is found and it is
valid (as indicated by the stored ebp different from 0), then the
target to which control is tranfered is defined by the lexically
stored ebp/address pair.  The code to which control is transfered is
responsible for adjusting esp to a particular value relative to ebp.

\section{Transfer of control to an exit point}

Whenever transfer of control to an exit point is initiated, the exit
point is first searched for.  If no valid exit point can be found, an
error is signaled.  If a valid exit point is found, the stack must
then be unwound.  First, the dynamic environment is traversed for any
intervening exit points, and they are marked as invalid by storing 0
as the stored ebp value.  The traversal stops when the stack frame of
the valid exit point is reached.  Unwinding now begins.  The dynamic
environment is traversed again and thunks in UNWIND-PROTECT entries
are executed.  The traversal again stops when the stack frame of the
valid exit point is reached.

\begin{figure}
\begin{center}
\inputfig{fig-x86-32-unwind-protect.pdf_t}
\end{center}
\caption{\label{fig-x86-32-unwind-protect}
\texttt{UNWIND-PROTECT} tag for the x86-32 backend.}
\end{figure}

\begin{figure}
\begin{center}
\inputfig{fig-x86-32-dynamic-binding.pdf_t}
\end{center}
\caption{\label{fig-x86-32-dynamic-binding}
Dynamic variable binding for the x86-32 backend.}
\end{figure}

\section{Argument parsing}

By \emph{argument parsing}, we mean the process of analyzing the
arguments that were passed to a function, and using those arguments to
initialize variables corresponding to the names of the parameters of
that function.

Because of the presence of \emph{optional arguments} and \emph{keyword
  arguments} that arbitrary forms to evaluate, argument parsing is a
fairly complicated process.  The process is further complicated by the
fact that some of the parameters of the function might be
\emph{special variables} rather than \emph{lexical variables}.

Conceptually, upon entry to a function, a new level of the lexical
runtime environment is created in order to hold the lexical variables
and temporaries of this function.  The argument-parsing process
consists of analyzing the arguments on the stack and filling in the
newly created level of the lexical runtime environment.  When this
process is done, the arguments are removed from the stack, leaving
only the four words that are always present in the stack frame, plus
zero, one, or more dynamic variable entries.
\seefig{fig-x86-32-dynamic-binding}

While ordinarily, the binding of a special variable is implemented as
a function call, passing a closure as an argument, luckily the full
implementation is not required when the special variable binding is
part of the argument-passing process.  The reason is that in all cases
where the stack might be unwound so that the binding is undone, then
the entire invocation of the function must be unwound as well.  Then,
the binding can be part of the current stack frame rather than
requiring its own stack frame.

Most of the time, there are lexical variables and temporaries that are
not closed over.  Such variables need not be allocated in a level of
the lexical runtime environment, but can be allocated on the stack.
In this case, the place on the stack that these variables will occupy
overlaps with the place that the initial arguments are passed.  For
that reason, the arguments are first moved, leaving a hole in the
stack frame corresponding to the number of lexical variables and
temporaries that are not closed over.  Then, in each step, argument
parsing will either initialize one of these variables, or one of the
variables in the level of the lexical runtime environment.  Finally,
the arguments are removed from the stack, leaving only the lexical
variables and temporaries that are not closed over.  However, things
are complicated by the fact that as part of the argument-parsing
process, entries for dynamic variables may have been created.
Removing the arguments, then, consists of moving those entries to
clobber the arguments on the stack, and adjusting the dynamic
environment word to reflect this move.

As a special case of the scenario in the previous paragraph,
\emph{all} lexical variables and temporaries can be allocated on the
stack.  Then, no level of the lexical runtime environment will be
created at all, and the hole in the stack frame will be big enough to
hold all such variables.

Also as a special case of the scenario in that same paragraph, it is
common that some \emph{prefix} of the parameters of the function
contains variables that are not closed over.  The variables in this
prefix are already in their ultimate places, so only the remaining
arguments need to be moved.

When a function takes only required parameters and none of those
parameters are closed over, then the argument-parsing process is
reduced to \emph{nothing}, and all that needs to be done upon function
entry is to allocate space in the stack frame for temporaries.

